\usepackage{etex,tabularx}
\newcolumntype{Y}{>{\centering\arraybackslash}X}
\usepackage{fontspec}
\IfFontExistsTF{FZShuSong-Z01}{
  \PassOptionsToPackage{ctex}{fontset=founder}
}{}
\usepackage[heading]{ctex}
\usepackage{manfnt}
\usepackage{indentfirst}
\usepackage{zhnumber}
\usepackage[T1]{fontenc}
\renewcommand\rmdefault{ptm}
\usepackage[centering,
           top=2.54cm,bottom=2.54cm,right=2.9cm,left=2.9cm,
           headsep=25pt,headheight=20pt]{geometry}
\IfFileExists{mtpro2.sty}{
  \usepackage[lite,zswash]{mtpro2}
}{}
\usepackage{amssymb}
\usepackage{amsmath,mathrsfs}
\usepackage{caption}

\usepackage{extarrows}
\usepackage{graphicx}

\setmainfont{Times New Roman}

\usepackage{pifont}
\usepackage{imakeidx}
\makeindex[
title={名词索引},
intoc=true,
columns=2,
columnsep=1cm,
columnseprule=true,
program=makeindex,
options={-s mkind.ist},
noautomatic=false
]
\indexsetup{
toclevel=chapter,
headers={名词索引}{名词索引},
othercode={
\renewcommand{\indexspace}{\smallskip}
}
}

\usepackage[hyperindex]{hyperref}
\hypersetup{bookmarksopen=true,bookmarksopenlevel=1,bookmarksnumbered=true,
  pdftitle={复变函数-史济怀},pdfauthor={史济怀 \& 刘太顺(向禹重排)},linktoc=all,CJKbookmarks=true,unicode,
  colorlinks,linkcolor=blue,citecolor=red,urlcolor=blue,anchorcolor=green}
\usepackage[Symbol]{upgreek}
\renewcommand{\pi}{\uppi}

\DeclareSymbolFont{ugmL}{OMX}{yhex}{m}{n}
\DeclareMathAccent{\wideparen}{\mathord}{ugmL}{"F3}

\IfFontExistsTF{FZShuSong-Z01}{
  \setCJKmainfont[BoldFont={FZHei-B01},ItalicFont={FZKai-Z03}]{FZShuSong-Z01}
}{}
\IfFontExistsTF{Microsoft YaHei}{
  \newCJKfontfamily\wryh{Microsoft YaHei}
}{
  \let\wryh\sffamily
}
\IfFontExistsTF{汉仪大宋简}{
  \newCJKfontfamily\hyds{汉仪大宋简}
}{
  \let\hyds\rmfamily
}

\defaultfontfeatures{Mapping=tex-text}
\XeTeXlinebreaklocale ”zh”
\XeTeXlinebreakskip = 0pt plus 1pt



\usepackage{tasks}
\settasks{
label = (\arabic*),
item-indent = 1.7em,
label-width = 0.5em,
label-offset = 1.2em,
column-sep=10pt,
label-align=left,
after-item-skip=0pt
}
\usepackage{multicol}

\everymath{\displaystyle}

\renewcommand{\le}{\leqslant}
\renewcommand{\ge}{\geqslant}

\allowdisplaybreaks[4]


\newcommand\OR{\overrightarrow}



\usepackage{fancyhdr}
\usepackage{tkz-euclide}
\usetikzlibrary{shapes.callouts,arrows.meta,calc,shadings}
\renewcommand{\Re}{\operatorname{\textrm{Re}}}
\renewcommand{\Im}{\operatorname{\textrm{Im}}}
\newcommand{\ii}{\textrm i}
\ctexset{punct=kaiming}
\renewcommand\thempfootnote{\ding{45}}
\newenvironment{note}{\par\CJKfamily{note}\noindent{\makebox[0pt][r]{\scriptsize\color{red!90}
\textdbend\quad}\textbf{提示:}}}{\par}



\usepackage{tocloft}
\renewcommand\cftchapfont{\hyds}
\renewcommand{\cftchapleader}{\cftdotfill{\cftdotsep}}
\ctexset {
  chapter = {
  beforeskip = 0pt,
  fixskip = true,
  format = \Huge\hyds,
  nameformat = \rule{\linewidth}{1bp}\par\bigskip\hfill\label{chap\thechapter}\chapternamebox,
  number = \arabic{chapter},
  aftername = \par\medskip,
  aftertitle = \par\bigskip\nointerlineskip\rule{\linewidth}{2bp}\par
},
  section = {
    titleformat+  = \label{sec\thesection}\raggedright
  },
  contentsname = \hyds{目 录}
}
\newcommand\chapternamebox[1]{%
\parbox{\ccwd}{\linespread{1}\selectfont\centering #1}}



\usepackage{enumitem}
\newenvironment{enum}{\begin{enumerate}[label=\textbf{\arabic*.},ref=\arabic*,left=0cm]}
{
\end{enumerate}}
\DeclareMathOperator{\ee}{\!\!\;\textrm e}
\newcommand{\MR}{\mathbb R}
\newcommand{\MC}{\mathbb C}
\newcommand{\MF}{\mathbb F}
\newcommand{\MZ}{\mathbb Z}
\newcommand{\MN}{\mathbb N}
\newcommand{\MCF}{\mathscr F}

\newenvironment{eenum}{\begin{enumerate}[label=(\arabic*),left=0.85cm]}
{
\end{enumerate}}
\newenvironment{enuma}{\begin{enumerate}[label=(\arabic*),leftmargin=0.71cm]}
{
\end{enumerate}}
\newcommand\quan[1]{
\tikz[baseline=(a.base)]\node(a)[inner sep=0.5pt,draw,circle]{$#1$};
}
\newcommand\closure[1]{%
{}\mkern2mu\overline{\mkern-2mu#1}
}
\renewcommand\bar{\closure}



\newcounter{definition}
\counterwithin{definition}{section}
\newcommand{\defname}{定义}
\newenvironment{definition}{\par%
  \refstepcounter{definition}%
  \textbf{\defname\thedefinition}\quad
}{\par}

\newcommand{\theoremname}{定理}
\newenvironment{theorem}[1][]{\par%
  \refstepcounter{definition}%
  \textbf{\theoremname\thedefinition}#1\quad
}{\par}

\newcommand{\propname}{命题}
\newenvironment{prop}{\par%
  \refstepcounter{definition}%
  \textbf{\propname\thedefinition}\quad
}{\par}

%\newcounter{example}
%\counterwithin{example}{section}
\newcommand{\examplename}{例}
\newenvironment{example}{\par%
  \refstepcounter{definition}%
  \textbf{\examplename\thedefinition}\quad
}{\par}

\newcommand{\lemmaname}{引理}
\newenvironment{lemma}[1][]{\par%
  \refstepcounter{definition}%
  \textbf{\lemmaname\thedefinition}#1\quad
}{\par}

\newcommand{\corname}{推论}
\newenvironment{corollary}{\par%
  \refstepcounter{definition}%
  \textbf{\corname\thedefinition}\quad
}{\par}

\newenvironment{xiti}{\par%
\centerline{\textbf{习\quad 题\quad \thesection}}
\begin{enum}
}{\end{enum}}

\usepackage[amsmath,thmmarks]{ntheorem}
{
\theoremstyle{nonumberplain}
\theoremheaderfont{\bfseries}
\theorembodyfont{\normalfont}
\theoremsymbol{\mbox{$\Box$}}
\newtheorem{proof}{\indent 证\,}
\newtheorem{solution}{\indent 解\,}
}

\setlist{itemsep=0pt}


\catcode`\。=\active
\newcommand{。}{.}
\newcommand\pp[2]{\frac{\partial #1}{\partial #2}}
\newcommand\ppp[2]{\frac{\partial^2 #1}{\partial #2^2}}
\newcommand\pppp[3]{\frac{\partial^2 #1}{\partial #2\partial #3}}
\newcommand\dd[2]{\frac{\textrm d#1}{\textrm d#2}}
\newcommand\ddd[2]{\frac{\textrm d^2#1}{\textrm d#2^2}}
\newcommand\dx{\textrm dx}
\newcommand\dy{\textrm dy}
\newcommand\dz{\textrm dz}
\newcommand\dzz{\textrm d\bar z}
\usepackage[subrefformat=parens]{subcaption}
\DeclareMathOperator{\Arg}{Arg}
\usepackage{tkz-euclide}

\catcode`\;=\active
\newcommand{;}{\text{;}}
\numberwithin{equation}{section}
\DeclareMathOperator\diam{diam}
\DeclareMathOperator{\Log}{Log}
\DeclareMathOperator{\supp}{supp}
\DeclareMathOperator{\Aut}{Aut}
\DeclareMathOperator{\SL}{SL}
\DeclareMathOperator*{\Res}{Res}
\newcommand\DD{D}
