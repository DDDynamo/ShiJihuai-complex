\setcounter{chapter}{1}
\chapter{全纯函数\label{chap2}}
\section{复变函数的导数\label{sec2.1}}
现在把实变函数中导数的概念推广到复变函数中来.
\begin{definition}
设$F:D\to\MC$是定义在域$D$上的函数,$z_0\in D$.如果极限
\begin{equation}\label{eq2.1.1}
 \lim_{z\to z_0}\frac{f(z)-f(z_0)}{z-z_0}
\end{equation}
存在,就说$f$在$z_0$处\textbf{复可微}\index{Q!全纯函数!复可微}或\textbf{可微}\index{Q!全纯函数!可微},这个极限称为$f$在$z_0$处的\textbf{导数}\index{Q!全纯函数!导数}或\textbf{微商}\index{Q!全纯函数!微商},记作$f'(z_0)$.如果$f$在$D$中每点都可微,就称$f$是域$D$中的\textbf{全纯函数}\index{Q!全纯函数}或\textbf{解析函数}\index{Q!全纯函数!解析函数}.如果$f$在$z_0$的一个邻域中全纯,就称$f$在$z_0$处全纯.
\end{definition}

设$f$在$z_0$处可微. 若记$\Delta z=z-z_0$,则 \eqref{eq2.1.1} 式可以写成
\[\lim_{\Delta z\to0}\frac{f(z_0+\Delta z)-f(z_0)}{\Delta z}=f'(z_0),\]
或者
\begin{equation}\label{eq2.1.2}
f(z_0+\Delta z)-f(z_0)=f'(z_0)\Delta z+o(|\Delta z|).
\end{equation}
由此即得$\lim_{\Delta z\to0}f(z_0+\Delta z)=f(z_0)$,这说明$f$在$z_0$处连续.我们已经证明了
\begin{prop}\label{prop2.1.2}
  若$f$在$z_0$处可微,则必在$z_0$处连续.
\end{prop}

但反过来不成立,即若$f$在$z_0$处连续,则$f$未必在$z_0$处可微.

\begin{example}
函数$f(z)=\bar z$在$\MC$中处处不可微.
\end{example}
\begin{proof}
对于任意$z\in\MC$,有
 \[\frac{f(z_0+\Delta z)-f(z_0)}{\Delta z}=\frac{\bar{z+\Delta z}-\bar z}{\Delta z}=\frac{\bar{\Delta z}}{\Delta z}.\]
 如果让$\Delta z$取实数,则$\frac{\bar{\Delta z}}{\Delta z}=1$;如果让$\Delta z$取纯虚数,则$\frac{\bar{\Delta z}}{\Delta z}=-1$.因此,当$\Delta z\to0$时上述极限不存在,因而在$\MC$中处处不可导.
\end{proof}

但容易看出这个函数在$\MC$中却是处处连续的,这是一个处处连续、处处不可微的例子.其实,在复变函数中这种例子很多,例如$f(z)=\Re z,f(z)=|z|$都是.但在实变函数中,要举一个这样的例子却是相当困难的.这说明在复变函数中可微的要求比实变函数中要强得多,因而得到的结论也强得多,这在以后的学习中将逐步揭示出来.

由于导数的定义在形式上与实变函数中一样,实变函数中导数的运算法则依然成立.即若$f$和$g$在域$D$中全纯,那么$f\pm g,fg$也在$D$中全纯,而且
\begin{align*}
&\big(f(z)\pm g(z)\big)'=f'(z)\pm g'(z),\\
&\big(f(z)g(z)\big)'=f'(z)g(z)+f(z)g'(z).
\end{align*}
如果对每一点$z\in D,g(z)\ne 0$,那么$\frac fg$也是$D$中的全纯函数,而且
\[\left(\frac{f(z)}{g(z)}\right)'=\frac{f'(z)g(z)-g'(z)f(z)}{\big(g(z)\big)^2}.\]
\begin{prop}
设$D_1,D_2$是$\MC$中的两个域,且
\begin{align*}
&f:D_1\to D_2,\\
&g:D_2\to \MC
\end{align*}
都是全纯函数,那么$h=g\circ f$是$D_1\to\MC$的全纯函数,而且$h'(z)=g'\big(f(z)\big)f'(z)$.这里,$g\circ f$记$f$和$g$的复合函数:$g\circ f(z)=g\big(f(z)\big)$.
\end{prop}

证明与实变函数的情形一样,留给读者作为练习.
\begin{xiti}
\item 研究下列函数的可微性:
\begin{tasks}(2)
  \task $f(z)=|z|$;
  \task $f(z)=|z|^2$;
  \task $f(z)=\Re z$;
  \task $f(z)=\arg z$;
  \task $f(z)$为常数.
\end{tasks}
\item 设$f$和$g$都在$z_0$处可微,且$f(z_0)=g(z_0)=0,g'(z_0)\ne0$.证明:
\[\lim_{z\to z_0}\frac{f(z)}{g(z)}=\frac{f'(z_0)}{g'(z_0)}.\]
\item 设$f$和$g$分别是域$D$和域$G$上的全纯函数,如果$f(D)\subset G$,那么$g\circ f$也是$D$上的全纯函数,而且
\[\big(g\circ f\big)'(z)=g'\big(f(z)\big)f'(z).\]
\item 设域$G$和域$D$关于实轴对称.证明:如果$f(z)$是$D$上的全纯函数,那么$\bar{f(\bar z)}$是$G$上的全纯函数.
\end{xiti}

\section{Cauchy-Riemann方程\label{sec2.2}}
现在讨论$f$在某点$z_0$可微的充分必要条件,为此先引入$f$在$z_0$处实可微的概念.
\begin{definition}
设$f(z)=u(x,y)+\ii v(x,y)$是定义在域$D$上的函数,$z_0=x_0+\ii y_0\in D$.我们说$f$在$z_0$处\textbf{实可微}\index{Q!全纯函数!实可微},是指$u$和$v$作为$x,y$的二元函数在$(x_0,y_0)$处可微.
\end{definition}

今设$f$在$z_0$处实可微,按定义,有
\begin{gather}
u(x_0+\Delta x,y_0+\Delta y)-u(x_0,y_0)=\pp ux(x_0,y_0)\Delta x+\pp uy(x_0,y_0)\Delta y+o(|\Delta z|),\label{eq2.2.1}\\
v(x_0+\Delta x,y_0+\Delta y)-v(x_0,y_0)=\pp vx(x_0,y_0)\Delta x+\pp vy(x_0,y_0)\Delta y+o(|\Delta z|)\label{eq2.2.2},
\end{gather}
这里,$|\Delta z|=\sqrt{(\Delta x)^2+(\Delta y)^2}$.于是
\begin{align*}
  f(z_0+\Delta z)-f(z_0)={}&u(x_0+\Delta x,y_0+\Delta y)-u(x_0,y_0)\\
  &+  \ii\big(v(x_0+\Delta x,y_0+\Delta y)-v(x_0,y_0)\big)\\
  ={}&\pp ux(x_0,y_0)\Delta x+\pp uy(x_0,y_0)\Delta y+o(|\Delta z|)\\
  &+\ii\bigg(\pp vx(x_0,y_0)\Delta x+\pp vy(x_0,y_0)\Delta y+o(|\Delta z|)\bigg)\\
  ={}&\bigg(\pp ux(x_0,y_0)+\ii\pp vx(x_0,y_0)\bigg)\Delta x\\
  &+\bigg(\pp uy(x_0,y_0)+\ii\pp vy(x_0,y_0)\bigg)\Delta y+o(|\Delta z|)\\
  ={}&\pp fx(x_0,y_0)\Delta x+\pp fy(x_0,y_0)\Delta y+o(|\Delta z|).
\end{align*}
把$\Delta x=\frac12(\Delta z+\bar{\Delta z}),\Delta y=\frac1{2\ii}(\Delta z-\bar{\Delta z})$代入上式,得
\begin{align*}
  f(z_0+\Delta z)-f(z_0)={}&\frac12\pp fx(x_0,y_0)(\Delta z+\bar{\Delta z})\\
  &-\frac{\ii}2\pp fy(x_0,y_0)(\Delta z-\bar{\Delta z})+o(|\Delta z|)\\
  ={}&\frac12\bigg(\pp {}x-\ii\pp{}y\bigg)f(x_0,y_0)\Delta z\\
  &+\frac12\bigg(\pp{}x+\ii\pp{}y\bigg)f(x_0,y_0)\bar{\Delta z}+o(|\Delta z|).
\end{align*}
引进算子
\begin{equation}\label{eq2.2.3}
\begin{gathered}
  \pp{}z=\frac12\bigg(\pp{}x-\ii\pp{}y\bigg),\\
  \pp{}{\bar z}=\frac12\bigg(\pp{}x+\ii\pp{}y\bigg),
\end{gathered}
\end{equation}
则上式可写为
\begin{equation}\label{eq2.2.4}
  f(z_0+\Delta z)-f(z_0)=\pp fz(z_0)\Delta z+\pp f{\bar z}(z_0)\bar{\Delta z}+o(|\Delta z|).
\end{equation}
容易看出,\eqref{eq2.2.4} 和 \eqref{eq2.2.1},\eqref{eq2.2.2} 两式等价.因而有下面的
\begin{prop}
设$f:D\to\MC$是定义在域$D$上的函数,$z_0\in D$,那么$f$在$z_0$处实可微的充分必要条件是  \eqref{eq2.2.4} 式成立,其中,$\pp{}z$是$\pp{}{\bar z}$和是由 \eqref{eq2.2.3} 式定义的算子.
\end{prop}

为什么要像 \eqref{eq2.2.3} 式那样来定义算子$\pp{}z$和$\pp{}{\bar z}$呢?这是因为如果把复变函数$f(z)$写成
\[f(x,y)=f\bigg(\frac{z+\bar z}2,-\ii\frac{z-\bar z}2\bigg),\]
把$z,\bar z$看成独立变量,分别对$z$和$\bar z$求偏导数,则得
\begin{gather*}
\pp fz=\pp fx\pp xz+\pp fy\pp yz=\frac12\bigg(\pp fx-\ii\pp fy\bigg),\\
\pp f{\bar z}=\pp fx\pp x{\bar z}+\pp fy\pp y{\bar z}=\frac12\bigg(\pp fx+\ii\pp fy\bigg).
\end{gather*}

这就是表达式 \eqref{eq2.2.3} 的来源.这说明在进行微分运算时,可以把$z,\bar z$交看成独立的变量.

现在很容易得到$f$在$z_0$处可微的条件了.
\begin{theorem}\label{thm2.2.3}
设$f$是定义在域$D$上的函数,$z_0\in D$,那么$f$在$z_0$处可微的充要条件是$f$在$z_0$处实可微且$\pp f{\bar z}(z_0)=0$.在可微的情况下,$f'(z_0)=\pp fz(z_0)$.
\end{theorem}
\begin{proof}
如果$f$在$z_0$处可微,由 \ref{sec2.1} 节的 \eqref{eq2.1.2} 式得
\[f(z_0+\Delta z)-f(z_0)=f'(z_0)\Delta z+o(|\Delta z|).\]
与 \eqref{eq2.2.4} 式比较就知道,$f$在$z_0$处是实可微的,而且$\pp f{\bar z}(z_0)=0,f'(z_0)=\pp fz(z_0)$.

反之,若$f$在$z_0$处实可微,且$\pp f{\bar z}(z_0)=0$,则由 \eqref{eq2.2.4} 得
\[f(z_0+\Delta z)-f(z_0)=f'(z_0)\Delta z+o(|\Delta z|).\]
由此即知$f$在$z_0$处可微,而且$f'(z_0)=\pp fz(z_0)$.
\end{proof}

$\pp f{\bar z}=0$称为\textbf{Cauchy-Riemann 方程}\index{G!公式!Cauchy-Riemann 方程},从这个方程可以得到$f$的实部和虚部应满足的条件. 设$f=u+\ii v$,则由 \eqref{eq2.2.3} 式得
\begin{align*}
  \pp f{\bar z}&=\pp u{\bar z}+\ii\pp v{\bar z}=\frac12\bigg(\pp ux+\ii \pp uy\bigg)
  +\frac{\ii}2\bigg(\pp vx+\ii\pp vy\bigg)\\
  &=\frac12\bigg(\pp ux-\pp vy\bigg)+\frac{\ii}2\bigg(\pp uy+\pp vx\bigg).
\end{align*}
因此,Cauchy-Riemann 方程$\pp f{\bar z}=0$就等价于
\begin{equation}\label{eq2.2.5}
\left\{\begin{aligned}
&\pp ux=\pp vy,\\
&\pp uy=-\pp vx.
\end{aligned}\right.
\end{equation}
这样,$f$在$z_0$处可微的条件可用它的实部和虚部表示为
\begin{theorem}\label{thm2.2.4}
设$f=u+\ii v$是定义在域$D$上的函数,$z_0=x_0+\ii y_0\in D$,那么$f$在$z_0$处可微的充要条件是$u(x,y),v(x,y)$在$(x_0,y_0)$处可微,且在$(x_0,y_0)$处满足
\begin{equation*}
\left\{\begin{aligned}
&\pp ux=\pp vy,\\
&\pp uy=-\pp vx.
\end{aligned}\right.
\end{equation*}
在可微的情况下,有
\begin{align*}
  f'(z_0)&=\pp ux+\ii \pp vx=\pp vy+\ii \pp vx\\
  &=\pp ux-\ii\pp uy=\pp vy-\ii \pp uy,
\end{align*}
这里的偏导数都在$(x_0,y_0)$处取值.
\end{theorem}

最后这个$f'(z_0)$的表达式是从$f'(z_0)=\pp fz(z_0)$和 Cauchy-Riemann 方程 \eqref{eq2.2.5} 得到的.

设$D$是$\MC$中的域,我们用$C(D)$记$D$上连续函数的全体,用$H(D)$记$D$上全纯函数的全体.命题 \ref{prop2.1.2} 告诉我们,$H(D)\subset C(D)$.

设$f=u+\ii v$,记$\pp fx=\pp ux+\ii\pp vx,\pp fy=\pp uy+\ii\pp vy$. 我们用$C^1(D)$记$\pp fx,\pp fy$在$D$上连续的$f$的全体. 由多元微积分的知识知道,对于任意$f\in C^1(D)$,$f$在$D$上实可微,从 \eqref{eq2.2.4} 式知道,$f$在$D$上连续,因而
\[C^1(D)\subset C(D).\]
用$C^k(D)$记在$D$上有$k$阶连续偏导数的函数的全体,$C^\infty(D)$记在$D$上有任意阶连续偏导数的函数的全体.以后将证明(定理 \ref{thm3.4.4}),域$D$上的全纯函数在$D$上有任意阶的连续偏导数,因而上面这些函数类有如下的包含关系:
\[H(D)\subset C^\infty(D)\subset C^k(D)\subset C^1(D)\subset C(D),\]
这里,$k$是大于$1$的自然数.

\begin{example}
研究函数$f(z)=z^n,n$是自然数.
\end{example}
\begin{solution}
  显然,$\pp f{\bar z}=0$,且$f$在整个平面上实可微的. 因而,$f$是$\MC$上的全纯函数,而且
  \[f'(z)=\pp fz=nz^{n-1}.\]
\end{solution}

\begin{example}
研究函数$f(z)=\ee^{-|z|^2}$.
\end{example}
\begin{solution}
把$f$写为$f(z)=\ee^{-z\bar z}$,于是$\pp f{\bar z}=-\ee^{-z\bar z}z$,它只有在$z=0$处才等于零.因此,$\ee^{-|z|^2}$只有在$z=0$处可微,它在任何点处都不是全纯的.但它对$x,y$有任意阶连续偏导数,所以它是$C^\infty(\MC)$中的函数.
\end{solution}

\begin{definition}\label{def2.2.7}
  设$u$是$D$上的实值函数,如果$u\in C^2(D)$,且对任意$z\in D$,有
  \[\Delta u(z)=\ppp{u(z)}{x}+\ppp{u(z)}{y}=0,\]
就称$u$是$D$中的\textbf{调和函数}\index{T!调和函数}. $\Delta =\ppp{}{x}+\ppp{}{y}$称为\textbf{Laplace算子}.\index{L!Laplace算子}
\end{definition}

\begin{prop}\label{prop2.2.8}
设$u\in C^2(D)$,那么$\Delta u=4\pppp uz{\bar z}$.
\end{prop}
\begin{proof}
由 \eqref{eq2.2.3} 式,有
\[\pp u{\bar z}=\frac12\bigg(\pp ux-\ii\pp uy\bigg),\]
所以
\begin{align*}
\pppp uz{\bar z}&=\pp {}z\bigg(\pp u{\bar z}\bigg)\\
&=\frac14\bigg[\pp{}x\bigg(\pp ux-\ii \pp uy\bigg)+\ii\pp{}y\bigg(\pp ux-\ii\pp uy\bigg)\bigg]\\
&=\frac14\bigg(\ppp ux+\ppp uy\bigg)=\frac14\Delta u.
\end{align*}
\end{proof}

由此便可得
\begin{theorem}\label{thm2.2.9}
  设$f=u+\ii v\in H(D)$,那么$u$和$v$都是$D$上的调和函数.
\end{theorem}
\begin{proof}
因为$f\in H(D)$,由 Cauchy-Riemann 方程,有
\[\pp f{\bar z}=0,\quad \pp{\bar f}z=0.\]
所以
\[\pppp fz{\bar z}=\pppp{\bar f}z{\bar z}=0.\]
于是,由$u=\frac12(f+\bar f)$即得
\[\Delta u=4\pppp uz{\bar z}=0.\]

同理可证$\Delta v=0$.
\end{proof}
\begin{definition}\label{def2.2.10}
设$u$和$v$是一对调和函数,如果他们还满足 Cauchy-Riemann 方程
\[\left\{\begin{aligned}
&\pp ux=\pp vy,\\
&\pp uy=-\pp vx,
\end{aligned}\right.\]
就称$v$为$u$的\textbf{共轭调和函数}.\index{T!调和函数!共轭调和函数}
\end{definition}

显然,全纯函数的实部和虚部就构成一对共轭调和函数.现在问,给定域$D$中的调和函数$u$,是否存在$u$的共轭调和函数$v$,使得$u+\ii v$成为$D$中的全纯函数?对于单连通域,答案是肯定的.
\begin{theorem}
设$u$是单连通域$D$上的调和函数,则必存在$u$的共轭调和函数$v$,使得$u+\ii v$是$D$上的全纯函数.
\end{theorem}
\begin{proof}
  因为$u$满足 Laplace 方程
  \[\ppp ux+\ppp uy=0,\]
若令$P=-\pp uy,Q=\pp ux$,则
\[\pp Qx=\ppp ux=-\ppp uy=\pp Py,\]
所以
\[P\dx+Q\dy=-\pp uy\dx+\pp ux\dy\]
是一个全微分,因而积分
\[\int_{(x_0,y_0)}^{(x,y)}-\pp uy\dx+\pp ux\dy\]
与路径无关. 令
\[v(x,y)=\int_{(x_0,y_0)}^{(x,y)}-\pp uy\dx+\pp ux\dy,\]
那么
\[\left\{\begin{aligned}
&\pp vx=-\pp uy,\\
&\pp vy=\pp ux.
\end{aligned}\right.\]
所以,$v$就是要求的$u$的共轭调和函数.
\end{proof}

关于调和函数的理论,我们将在第 \ref{chap8} 章中作进一步的讨论.
\begin{xiti}\hypertarget{xiti2.2}{}
\item \hypertarget{xiti2.2.1}{}设$D$是$\MC$中的域,$f\in H(D)$. 如果对每一个$z\in D$,都有$f'(z)=0$,证明$f$是一常数.
\item 设$f\in H(D)$,并且满足下列条件之一:
\begin{enuma}
  \item $\Re f(z)$是常数;
  \item $\Im f(z)$是常数;
  \item $|f(z)|$是常数;
  \item $\arg f(z)$是常数;
  \item $\Re f(z)=\big(\Im f(z)\big)^2,z\in D$,
\end{enuma}
那么$f$是一常数.
\item 设$z=x+\ii y$,证明$f(z)=\sqrt{xy}$在$z=0$处满足 Cauchy-Riemann 方程,但$f$在$z=0$处不可微.
\item \hypertarget{xiti2.2.4}{}设$z=r(\cos\theta+\ii\sin\theta),f(z)=u(r,\theta)+\ii v(r,\theta)$,证明 Cauchy-Riemann 方程为
    \[\left\{\begin{aligned}
      &\pp ur=\frac1r\pp v{\theta},\\
      &\pp vr=-\frac1r\pp u{\theta}.
    \end{aligned}\right.\]
\item 设$z=r(\cos\theta+\ii\sin\theta)$.证明:
\begin{gather*}
  \pp f{\bar z}=\frac12\ee^{\ii\theta}\bigg(\pp fr+\frac{\ii}r\pp f{\theta}\bigg),\\
  \pp fz=\frac12\ee^{-\ii\theta}\bigg(\pp fr-\frac{\ii}r\pp f{\theta}\bigg).
\end{gather*}
\item 设$\boldsymbol s$和$\boldsymbol n$是两个平面向量,将$\boldsymbol s$按逆时针方向旋转$\frac\pi2$后即为$\boldsymbol n$.如果$f=u+\ii v$是全纯函数,证明
  \[\left\{\begin{aligned}
  &\pp u{\boldsymbol s}=\pp v{\boldsymbol n},\\
  &\pp u{\boldsymbol n}=-\pp v{\boldsymbol s}.
  \end{aligned}\right.\]
\item 设$D$是$\MC$中的域,$f\in C^2(D)$.证明:对每个$z\in D$,有
\[\pppp fz{\bar z}(z)=\pppp f{\bar z}z(z).\]
\item 设$D$是$\MC$中的域,$f\in H(D),f$在$D$中不取零值.证明:对任意$p>0$,有
\[\bigg(\ppp{}x+\ppp{}y\bigg)|f(z)|^p=p^2|f(z)|^{p-2}|f'(z)|^2.\]
\item 设$D$是$\MC$中的域,$f=u+\ii v\in C^1(D)$.证明:
\[\begin{vmatrix}
\pp ux&\pp uy\\\pp vx&\pp vy
\end{vmatrix}=\bigg|\pp fz\bigg|^2-\bigg|\pp f{\bar z}\bigg|^2.\]
特别地,当$f\in H(D)$时,有
\[\begin{vmatrix}
\pp ux&\pp uy\\\pp vx&\pp vy
\end{vmatrix}=|f'|^2.\]
给出上面等式的几何意义.
\item 设$D$是域,$n$是自然数. 证明:$f\in C^n(D)$当且仅当$\frac{\partial^nf}{\partial z^k\partial\bar z^{n-k}}$在$D$上连续,$0\le k\le n$.
\item 设$D$是域,$f:D\to\MC\backslash(-\infty,0]$是非常数的全纯函数,则$\log |f(z)|$和$\arg f(z)$是$D$上的调和函数,而$|f(z)|$不是$D$上的调和函数.
\item 设$D,G$是域,$\varphi:D\to G$是全纯函数。证明:若$u$是$G$上的调和函数.则$u\circ f$是$D$上的调和函数.
\item 设$u$是域$D$上的调和函数,$\varphi$是$u(D)$上的实函数.证明:
$\varphi\circ u$是$D$上的调和函数当且仅当$\varphi$是线性函数.
\item 设$D,G$是域,$U\in C^2(G),\varphi\in H(D)$,并且$\varphi(D)\subset G$.证明:$\Delta (u\circ \varphi)=\Delta u|\varphi'|^2$.
\item 举例说明:存在$B(0,1)\backslash\{0\}$上的调和函数,它不是$B(0,1)\backslash\{0\}$上全纯函数的实部.
\item 设$f=u+\ii v,z_0=x_0+\ii y_0$.证明:
\begin{enuma}
  \item 如果极限$\lim_{z\to z_0}\Re\frac{f(z)-f(z_0)}{z-z_0}$存在,那么$\pp ux(x_0,y_0)$和$\pp vy(x_0,y_0)$存在,并且相等;
  \item 如果极限$\lim_{z\to z_0}\Im\frac{f(z)-f(z_0)}{z-z_0}$存在,那么$\pp uy(x_0,y_0)$和$\pp vx(x_0,y_0)$存在,而且
      \[\pp uy(x_0,y_0)=-\pp vx(x_0,y_0).\]
\end{enuma}
\item 证明:若$f(z)$在$z_0$处实可微,并且$\lim_{z\to z_0}\bigg|\frac{f(z)-f(z_0)}{z-z_0}\bigg|$存在,则$f(z)$和$\bar{f(z)}$必有一个在$z_0$处可微.
\item 证明:若$u(x,y)$是$x,y$的调和多项式,则
\[f(z)=2u\bigg(\frac z2,\frac z{2\ii}\bigg)-u(0,0)\]
是$\MC$上的全纯函数,并且对任意$z=x+\ii y\in\MC,\Re f(z)=u(x,y)$.
\end{xiti}

\section{导数的几何意义\label{sec2.3}}
设$f$是域$D$上的连续函数,$z_0\in D$,如果$f$在$z_0$处全纯,且$f'(z_0)\ne0$,我们来讨论$f'(z_0)$这个复数的几何意义.

过$z_0$作一条光滑曲线$\gamma$,它的方程为
\[z=\gamma(t),a\le t\le b.\]
设$\gamma(a)=z_0$,且$\gamma'(a)\ne0$.前面说过,$\gamma$在点$z_0$处的切线与正实轴的夹角为$\Arg\gamma'(a)$.设$w=f(z)$把曲线$\gamma$映为$\sigma$,它的方程为\[w=\sigma(t)=f\big(\gamma(t)\big),a\le t\le b.\]
由于$\sigma'(a)=f'\big(\gamma(a)\big)\gamma'(a)=f'(z_0)\gamma'(a)\ne0$,所以$\sigma$在$w_0=f(z_0)$处的切线与正实轴的夹角为
\[\Arg\sigma'(a)=\Arg f'(z_0)+\Arg \gamma'(a),\]
或者写为
\begin{equation}\label{eq2.3.1}
\Arg\sigma'(a)-\Arg \gamma'(a)=\Arg f'(z_0).
\end{equation}
这说明像曲线$\sigma$在$w_0$处的切线与正实轴的夹角与原曲线$\gamma$在$z_0$处的切线与正实轴的夹角之差总是$\Arg f'(z_0)$,而与曲线$\gamma$无关.
$\Arg f'(z_0)$就称为映射$w=f(z)$在点$z_0$处的转动角.这一事实导致下面的重要结果:

如果过$z_0$点作两条光滑曲线$\gamma_1,\gamma_2$,它们的方程分别为
\[z=\gamma_1(t),a\le t\le b\]
和
\[z=\gamma_2(t),a\le t\le b,\]
且$\gamma_1(a)=\gamma_2(a)=z_0$(图 \ref{fig2.1}\subref{fig2.1a}). 映射$w=f(z)$把它们分别映为过$w_0$点的两条光滑曲线$\sigma_1$和$\sigma_2$(图 \ref{fig2.1}\subref{fig2.1b}),它们的方程分别为
\[w=\sigma_1(t)=f\big(\gamma_1(t)\big),a\le t\le b.\]
和
\[w=\sigma_2(t)=f\big(\gamma_2(t)\big),a\le t\le b.\]
由 \eqref{eq2.3.1} 式可得
\[\Arg\sigma_1'(a)-\Arg\gamma_1'(a)=\Arg f'(z_0)=\Arg\sigma_2'(a)-\Arg\gamma_2'(a),\]
即
\begin{equation}\label{eq2.3.2}
\Arg\sigma_2'(a)-\Arg\sigma_1'(a)=\Arg\gamma_2'(a)-\Arg\gamma_1'(a).
\end{equation}
\begin{figure}[!ht]
\centering
\subcaptionbox{\label{fig2.1a}}[0.48\textwidth]
{
\begin{tikzpicture}[thick,>=Stealth]
\draw[->](-0.5,0)--(0,0)node[below left]{$O$}--(4,0);
\draw[->](0,-0.5)--(0,3.8);
\draw(0.6,0.6)node[below left,inner sep=0pt]{$z_0$}--(4,0.6);
\draw(0.6,0.6)arc(-90:-30:3)node[above]{$\gamma_1$};
\draw(1.1,0.6)arc(0:25:0.5);
\begin{scope}[rotate around={25:(0.6,0.6)}]
\draw(0.6,0.6)--(4,0.6);
\draw(0.6,0.6)arc(-90:-30:3)node[above]{$\gamma_2$};
\end{scope}
\end{tikzpicture}
}
\subcaptionbox{\label{fig2.1b}}[0.48\textwidth]
{
\begin{tikzpicture}[thick,>=Stealth]
\draw[->](-0.5,0)--(0,0)node[below left]{$O$}--(4,0);
\draw[->](0,-0.5)--(0,3.8);
\begin{scope}[rotate around={10:(0.6,0.6)}]
\draw(0.6,0.6)node[below left,inner sep=0pt]{$w_0$}--(4,0.6);
\draw(0.6,0.6)arc(-90:-30:3)node[above]{$\sigma_1$};
\draw(1.1,0.6)arc(0:25:0.5);
\begin{scope}[rotate around={25:(0.6,0.6)}]
\draw(0.6,0.6)--(4,0.6);
\draw(0.6,0.6)arc(-90:-30:3)node[above]{$\sigma_2$};
\end{scope}
\end{scope}
\end{tikzpicture}
}
\caption{\label{fig2.1}}
\end{figure}
上式左端是曲线$\sigma_1$和$\sigma_2$在$w_0$处的夹角(两条曲线在某点的夹角定义为这两条曲线在该点的切线的夹角),右端是曲线$\gamma_1$和$\gamma_2$在$z_0$处的夹角. \eqref{eq2.3.2} 式说明,如果$f'(z_0)\ne0$,那么在映射$w=f(z)$的作用下,过$z_0$点的任意两条光滑曲线的夹角的大小与旋转方向都是保持不变的.我们把具有这种性质的映射称为在$z_0$点是\textbf{保角的}\index{B!变换!保角变换}.这样,我们已经证明了
\begin{theorem}
全纯函数在其导数不为零的点处是保角的.
\end{theorem}

再来看导数的模的几何意义.和刚才一样,过$z_0$点作曲线$\gamma$,它在映射$f$下的像为$\sigma$(图 \ref{fig2.2}).由于
\[\lim_{z\to z_0}=\frac{f(z)-f(z_0)}{z-z_0}=f'(z_0),\]
所以,当$z$沿着$\gamma$趋于$z_0$时,有
\[\lim_{z\to z_0}\frac{|f(z)-f(z_0)|}{|z-z_0|}=\lim_{z\to z_0}\frac{|w-w_0|}{|z-z_0|}=|f'(z_0)|.\]
这说明像点之间的距离与原像之间的距离之比只与$z_0$有关,而与曲线$\gamma$无关.称$|f'(z_0)|$为$f$在$z_0$处的\textbf{伸缩率}\index{S!伸缩率}.
\begin{figure}[!ht]
\centering
\begin{tikzpicture}[thick,>=Stealth]
\draw[->](-0.5,0)--(0,0)node[below left]{$O$}--(4,0);
\draw[->](0,-0.5)--(0,3.8);
\draw(0.6,0.6)arc(180:170:2)coordinate(A)node[left]{$z_0$}
arc(170:100:2)coordinate(B)node[above]{$z$}arc(100:90:2);
\draw(A)--node[above left,inner sep=2mm]{$\gamma$}(B);
\draw[->](3.9,2)--node[above]{$w=f(z)$}(5.7,2);
\begin{scope}[xshift=6.2cm]
\draw[->](-0.5,0)--(0,0)node[below left]{$O$}--(4,0);
\draw[->](0,-0.5)--(0,3.8);
\draw[scale=1.4](0.6,0.6)arc(180:170:2)coordinate(C)node[left]{$w_0$}
arc(170:100:2)coordinate(D)node[above]{$w$}arc(100:90:2);
\draw(C)--node[above left,inner sep=3.4mm]{$\sigma$}(D);
\end{scope}
\end{tikzpicture}
\caption{\label{fig2.2}}
\end{figure}

综合导数辐角和模的几何意义,我们看到:如果$f'(z_0)\ne0$,在$z_0$的邻域中,作一个以$z_0$为顶点的小三角形,这个小三角形被$f$映射为一个曲边三角形,它的微分三角形和原来的小三角形相似(图 \ref{fig2.3}).因此,我们把这样一个映射称为\textbf{共形映射}\index{F!复变函数!共形映射}.
\begin{figure}[!ht]
\centering
\subcaptionbox{\label{fig2.3a}}[0.48\textwidth]
{
\begin{tikzpicture}[thick,>=Stealth]
\draw[->](-0.5,0)--(0,0)node[below left]{$O$}--(4,0);
\draw[->](0,-0.5)--(0,3.8);
\draw(0.6,0.9)node[below left,inner sep=1pt]{$z_0$}--node[below]{$|\textrm dz|$}(2.4,0.9)--(2.4,2.3)--cycle;
\draw(0.9,0.9)arc(0:40:0.3);
\node at(1.05,1.07){$\theta$};
\end{tikzpicture}
}
\subcaptionbox{\label{fig2.3b}}[0.48\textwidth]
{
\begin{tikzpicture}[thick,>=Stealth]
\draw[->](-0.5,0)--(0,0)node[below left]{$O$}--(4,0);
\draw[->](0,-0.5)--(0,3.8);
\begin{scope}[scale=1.1]
\draw(0.6,0.9)node[below left,inner sep=1pt]{$w_0$}--node[below]{$|\textrm df|$}(2.4,0.9)--(2.4,2.3)--cycle;
\draw(0.9,0.9)arc(0:40:0.3);
\node at(1.1,1.12){$\theta$};
\draw[densely dashed](2.4,2.3)--(2.4,3.2);
\draw(0.6,0.9)..controls(0.6,0.9)and(1.8,1.8)..(2.4,3.2);
%\draw(0.6,0.9)..controls(0.6,0.9)and(1.8,0.8)..(2.8,1.6);
\draw plot[smooth]coordinates{(0.6,0.9)(1.9,1.1)(2.8,1.6)(2.4,3.2)};
\draw[densely dashed](2.4,0.9)--(2.8,1.57);
\end{scope}
\end{tikzpicture}
}
\caption{\label{fig2.3}}

\end{figure}
\begin{xiti}
\item 求映射$w=\frac{z-\ii}{z+\ii}$在$z_1=-1$和$z_2=\ii$处的转动角和伸缩率.
\item 设$f$是域$D$上的全纯函数,且$f'(z)$在$D$上不取零值. 试证:
\begin{enuma}
  \item 对每一个$u_0+\ii v_0\in f(D)$,曲线$\Re f(z)=u_0$和曲线$\Im f(z)=v_0$正交;
  \item 对每一个$r_0\ee^{\ii\theta_0}\in f(D)\backslash\{0\},-\pi<\theta_0\le\pi$,曲线$|f(z)|=r_0$与曲线$\arg f(z)=\theta_0$正交.
\end{enuma}
\item 设$f$在$B(0,1)\cup\{1\}$上全纯,并且
\[f\big(B(0,1)\big)\subset B(0,1),f(1)=1,\]
证明$f'(1)\ge0$.

(\textbf{提示}:在$f'(1)\ne0$的情形下,证明$\arg f'(1)=0$.)
\item 设$f\in H\big(B(0,1)\big)$,如果存在$z_0\in B(0,1)\backslash\{0\}$,使得$f(z_0)\ne0,f'(z_0)\ne0$,且$|f(z_0)|=\max_{|z|\le |z_0|}|f(z)|$,那么
    \[\frac{z_0f'(z_0)}{f(z_0)}>0.\]
\end{xiti}

\section{初等全纯函数\label{sec2.4}}
在讨论一般的全纯函数理论之前,先介绍几个初等的全纯函数.在微积分中,我们把幂函数、指数函数及其反函数对数函数、三角函数及其反函数反三角函数这三类函数叫做基本初等函数,由这些基本初等函数经过有限次的加、减、乘、除以及复合运算所得的函数称为初等函数.其实,幂函数$x^\mu$也可通过指数函数和对数函数复合而得:$x^\mu=\ee^{\mu\log x}$,但三角函数和指数函数没有直接的关系.因此,基本初等函数实际上只有指数函数和三角函数以及它们各自的反函数两类.下面我们将要看到,在复数域中,三角函数是可以用指数函数来表示的.因此,在复数域中,基本初等函数就只有指数函数及其反函数这一类.我们的讨论当然也从指数函数开始.

\subsection{指数函数}
设$z=x+\ii y$,如何定义复变数的指数函数$\ee^{z}$呢?考虑的原则有两条:第一,因为实变数的指数函数$\ee^x$在实轴上每点都有导数,所以我们要求$\ee^z$在平面$\MC$上每点都可导,即$\ee^z$是$\MC$上的全纯函数.第二,当$z=x$时,它和实变数的指数函数相一致.

下面的讨论对我们会有些启发.在微积分中我们已经知道,对任意实数$t$,有
\begin{align*}
&\ee^t=\sum_{n=0}^\infty \frac{t^n}{n!},\\
&\cos t=\sum_{n=0}^\infty(-1)^n\frac{t^{2n}}{(2n)!},\\
&\sin t=\sum_{n=0}^\infty(-1)^n\frac{t^{2n+1}}{(2n+1)!}.
\end{align*}
用$t=\ii y$代入$\ee^t$的展开式中,得
\begin{align*}
\ee^{\ii y}&=\sum_{n=0}^\infty\frac{(\ii y)^n}{n!}=\sum_{k=0}^\infty\frac{(\ii y)^{2k}}{(2k)!}+\sum_{k=0}^\infty\frac{(\ii y)^{2k+1}}{(2k+1)!}\\
&=\sum_{k=0}^\infty(-1)^k\frac{y^{2k}}{(2k)!}+
\ii\sum_{k=0}^\infty(-1)^k\frac{y^{2k+1}}{(2k+1)!}\\
&=\cos y+\ii\sin y.
\end{align*}
这个等式通常称为\textbf{Euler公式}\index{G!公式!Euler 公式},它启发我们给出$\ee^z$的下列定义:

设$z=x+\ii y$,定义
\[\ee^z=\ee^x(\cos y+\ii \sin y).\]
这样定义的指数函数有许多与实变数指数函数类似的性质,但也产生了一些新的性质:
\begin{eenum}\parindent=2\ccwd
\item $\ee^z$是$\MC$上的全纯函数,而且
  \[(\ee^z)'=\ee^z.\]

  $\ee^z$在$\MC$上每点实可微是显然的, 今验证它满足 Cauchy-Riemann 方程. 因为$u(x,y)=\ee^x\cos y,v(x,y)=\ee^x\sin y$,所以
  \begin{align*}
  &\pp ux=\ee^x\cos y=\pp vy,\\
  &\pp uy=-\ee^x\sin y=-\pp vx.
  \end{align*}
故由定理 \ref{thm2.2.4},$\ee^z$在$\MC$上全纯,而且
\[(\ee^z)'=\pp ux+\ii \pp vx=\ee^x\cos y+\ii \ee^x\sin y=\ee^z.\]

\item 当$z=x$时,即$y=0$,因而有$\ee^z=\ee^x$;当$z=\ii y$时,$\ee^{\ii y}=\cos y+\ii\sin y$.这样,复数的三角标表示$z=r(\cos\theta+\ii\sin\theta)$就可简单地写为$z=r\ee^{\ii\theta}$.

\item 对于任意$z\in \MC.\ee^z\ne0$. 这是因为
\[|\ee^z|=\ee^x>0.\]

\item 对于任意$z_1,z_2$,有
\[\ee^{z_1}\ee^{z_2}=\ee^{z_1+z_2}.\]

设$z_1=x_1+\ii y_1,z_2=x_2+\ii y_2$,直接计算即得
\begin{align*}
\ee^{z_1}\ee^{z_2}&=\ee^{x_1}(\cos y_1+\ii\sin y_1)\ee^{x_2}(\cos y_2+\ii \sin y_2)\\
&=\ee^{x_1+x_2}[\cos(y_1+y_2)+\ii\sin(y_1+y_2)]\\
&=\ee^{z_1+z_2}.
\end{align*}

\item $\ee^z$是以$2\pi\ii$为周期的周期函数,这是实变数指数函数$\ee^x$所没有的性质.证明当然很简单:
\begin{align*}
\ee^{z+2\pi\ii}&=\ee^{x+\ii(y+2\pi\ii)}\\
&=\ee^x[\cos(y+2\pi)+\ii\sin(y+2\pi)]=\ee^z.
\end{align*}
\end{eenum}

下面来研究$w=\ee^z$的映射性质. 先给出
\begin{definition}\label{def2.4.1}
设$f:D\to\MC$是一个复变函数,如果对域$D$中任意两点$z_1,z_2$($z_1\ne z_2$),必有$f(z_1)\ne f(z_2)$,就称$f$在$D$中是\textbf{单叶}\index{D!单叶函数}的,$D$称为$f$的\textbf{单叶性域}.\index{Y!域!单叶性域}.
\end{definition}

如果$f$在$D$中是单叶的,$f(D)=G$,那么$f$是$D$到$G$之上的一一映射.

现在来求$w=\ee^z$的单叶性域.如果$z_1=x_1+\ii y_1,z_2=x_2+\ii y_2$使得$\ee^{z_1}=\ee^{z_2} $,即$\ee^{x_1}\ee^{\ii y_1}=\ee^{x_2}\ee^{\ii y_2}$,那么$x_1=x_2,y_1=y_2+2k\pi,k$是任意整数,也即$z_1-z_2=2k\pi\ii$.这就是说,凡是不包含满足条件$z_1-z_2=2k\pi\ii$的$z_1,z_2$的域都是$w=\ee^z$的单叶性域.例如,域
\[\{z=x+\ii y:2k\pi<y<2(k+1)\pi\},k=0,\pm1,\cdots\]
都是$\ee^z$的单叶性域,它是平行于实轴、宽度为$2\pi$的带状域.由于$\ee^z$是以$2\pi\ii$为周期的函数,我们只要弄清$\ee^z$在域$\{z=x+\ii y:0<y<2\pi\}$中的映射性质,那么在其他带状域中的性质是一样的.

现在我们来研究$w=\ee^z$把平行于实轴的直线$\Im z=y_0$变成什么.这条直线上的点的方程为
\[z=x+\ii y_0,-\infty<x<\infty,\]
所以
\[w=\ee^z=\ee^x\ee^{\ii y_0}.\]
这是一条从原点出发的半射线,它与实轴正方向的夹角是$y_0$(图 \ref{fig2.4}).当$y_0$从$0$变到$2\pi$时,这条半射线的辐角也从$0$变到$2\pi$.因此,$w=\ee^z$把带状域$\{z=x+\ii y:0<y<2\pi\}$变成全平面除掉正实轴的域$\MC\backslash\{z:z\ge0\}$,直线$\Im z=0$变成正实轴的上岸,直线$\Im z=2\pi$变成正实轴的下岸;带状域$\{z=x+\ii y:0<y<\pi\}$变成上半平面,带状域$\{z=x+\ii y:\pi<y<2\pi\}$变成下半平面.一般来说,$w=\ee^z$把带状域$\{z=x+\ii y:\alpha<y<\beta,0<\alpha<\beta\le2\pi\}$变成角状域$\alpha<\arg w<\beta$.
\begin{figure}[!ht]
\centering
\subcaptionbox{\label{fig2.4a}}[0.48\textwidth]
{
\begin{tikzpicture}[thick,>=Stealth]
\draw[->](-2.1,0)--(0,0)node[below]{$O$}--(2.1,0);
\draw[->](0,0)--(0,4);
\draw(-2,3)--(0,3)node[above right]{$2\pi\textrm i$}--(2,3);
\draw(-2,1)--(0,1)node[above right]{$y_0\textrm i$}--(2,1);
\end{tikzpicture}
}
\subcaptionbox{\label{fig2.4b}}[0.48\textwidth]
{
\begin{tikzpicture}[thick,>=Stealth]
\draw[->](-2.1,0)--(0,0)node[below]{$O$}--(2.1,0);
\draw[->](0,0)--(0,4);
\draw(0,0)--(130:3);
\draw(0.5,0)arc(0:130:0.5);
\node at(70:0.7){$y_0$};
\end{tikzpicture}
}
\caption{\label{fig2.4}}
\end{figure}
\subsection{对数函数}
对于给定的$z\in \MC$,满足方程$\ee^w=z$的$w$称为$z$的对数,记为$w=\Log z$.现在给出由$z$计算$w$的公式.设$z=r\ee^{\ii\theta},w=u+\ii v$,则$\ee^{u+\ii v}=\ee^{\ii\theta}$,因而$\ee^u=r,v=\theta+2k\pi$.于是
\[\Log z=\log|z|+\ii\arg z+2k\pi\ii=\log|z|+\ii\Arg z.\]
所以,$\Log z$是一个多值函数,它的多值性是由$z$的辐角$\Arg z$的多值性产生的.对多值函数来说,一个重要的问题是:在什么样的域中,从这个多值函数中能取出单值的全纯的分支?对此,我们有
\begin{theorem}\label{thm2.4.2}
如果$D$是不包含原点和无穷远点的单连通域,则必在$D$上存在无穷多个单值全纯函数$\varphi_k,k=0,\pm1,\cdots$,使得在$D$上成立
\[\ee^{\varphi_k(z)}=z,k=0,\pm1,\cdots;\]
而且对每一个$k$,有$\varphi_k'(z)=\frac1z$.其中的每一个$\varphi_k$都称为$\Log z$在$D$上的\textbf{单值全纯分支}.\index{D!单值全纯分支}
\end{theorem}
\begin{proof}
对给定的$z$,选定它的辐角$\theta=\theta_0+2k_0\pi$,这里,$\theta_0$是$z$的辐角的主值,即$\theta_0=\arg z$,$k_0$是任意一个给定的整数.在$D$上定义
\[\varphi_{k_0}(z)=\log|z|+\ii(\theta_0+2k_0\pi)=\log r+\ii\theta,\]
这时,$u=\log r,v=\theta$.容易验证这时有
\[\pp ur=\frac1r\pp v\theta,\quad \pp u\theta=-r\pp vr,\]
因此由习题 \hyperlink{xiti2.2}{2.2} 的第 \hyperlink{xiti2.2.4}{4} 题知道,$\varphi_{k_0}$是$D$上的全纯函数,而且
\[\varphi_{k_0}'(z)=\frac rz\bigg(\pp ur+\ii\pp vr\bigg)=\frac1z.\]
此外,
\[\ee^{\varphi_k(z)}=\ee^{\log|z|+\ii(\theta_0+2k_0\pi)}=|z|\ee^{\ii\theta_0}=z,\]
对每一点$z\in D$成立.
\end{proof}

现在说明为什么要求$D$不包含原点和无穷远点.如果$D$包含原点,那么$D$中就包含绕原点$z=0$的简单闭曲线$\gamma$,当$z$从$\gamma$上的一点$z_0$沿$\gamma$的正方向(即反时针方向)回到$z_0$时,$z$的辐角增加了$2\pi$,$\varphi_{k_0}$的值从$\varphi_{k_0}(z_0)$连续地变为$\varphi_{k_0+1}(z_0)$,而不再回到原来的值$\varphi_{k_0}(z_0)$.因此,在这样的域中就不可能从$\Log z$中分出单值的全纯分支.因为$D$内任意一条绕原点的简单闭曲线也可以看作是绕无穷远点的简单闭曲线,因此$D$也不能包含无穷远点.

一般来说,我们有下面的
\begin{definition}\label{def2.4.3}
如果当$z$沿着$z_0$的充分小邻域中的任意简单闭曲线绕一圈时,多值函数的值就从一支变到另一支,那么称$z_0$为该多值函数的一个\textbf{支点}\index{Z!支点}.
\end{definition}

以对数函数为例,$z=0$和$z=\infty$便是$\Log z$的支点.

现在讨论$\Log z$的映射性质.根据定理 \ref{thm2.4.2},我们取$D$为$\MC$除去负实轴后所得的域,它是不包含原点和无穷远点的单连通域,因而可以分出无穷多个单值的全纯分支.我们把$k_0=0$的那一支称为它的\textbf{主支}\index{Z!主支},这时取$\Arg z$的主值为$-\pi<\arg z<\pi$,于是
\[w=\varphi_0(z)=\log|z|+\ii\arg z\]
把$D$单叶地映为带状域$-\pi<\Im w<\pi$.其他各分支,例如$w=\varphi_k(z)=\log|z|+\ii(\arg z+2k\pi)$,就把$D$单叶地映为带状域$(2k-1)\pi<\Im w<(2k+1)\pi$.一般来说,$w=\varphi_0(z)$把角状域$-\pi\le\alpha<\arg z<\beta\le\pi$单叶地映为带状域$\alpha<\Im w<\beta$. 今后,我们就把$\Log z$的主支$\varphi_0(z)$记为$\log z$.

有时,为了方便起见,也可把$\MC$去掉正实轴以后的域取为$D$,它同样是不包含原点和无穷远点的单连通域,但这时辐角的主值范围应取为$0<\arg z<2\pi$. $\Log z$的主支是
\[\log z=\log|z|+\ii\arg z,\quad 0<\arg z<2\pi,\]
它把$D$单叶地映为带状域$0<\Im w<2\pi$.

一般来说,还可以用一条从原点出发并伸向无穷远的曲线代替上面的负实轴或正实轴,这样得到的域$D$同样满足定理 \ref{thm2.4.2} 的条件.我们把这种从原点出发并伸向无穷远的曲线叫做\textbf{割线}\index{G!割线}.通常,为了便于表达出$\Log z$的单值分支$\varphi_k(z)$,常取从原点出发的一条射线作为割线,特别是取负实轴或正实轴.

\subsection{幂函数}
$w=z^\mu$称为幂函数,这里,$\mu=a+b\ii$是一个复数. 我们分几种情形来讨论.

(1) $\mu=n$,{\kaishu 是一个自然数.}

按导数的定义,可以直接算出
\[(z^n)'=nz^{n-1}.\]
所以,$w=z^n$在$\MC$上每点都是全纯的, 一般地,有
\begin{definition}\label{def2.4.4}
  在$\MC$上每点都全纯的函数称为\textbf{整函数}.\index{Q!全纯函数!整函数}
\end{definition}

所以,$w=z^n$是一个整函数.由于它的导数除原点外都不为零,因此除原点外它是一个保角变换,保角性在原点不成立.考虑从原点出发的射线,它与正实轴的夹角为$\theta$,这条射线的方程可写为$\arg z=\theta$.由于$w=z^n$,所以
\[\arg w=n\arg z=n\theta.\]
这就是说,这条射线的像也是一条过原点的射线,但它与正实轴的夹角是$n\theta$,已经比原来的夹角扩大了$n$倍.这一事实在作具体变换时却很有用.例如,$w=z^2$能把第一象限变成上半平面,$w=z^3$能把角状域$\bigg\{z:0<\arg z<\frac\pi3\bigg\}$变成上半平面,等等.

现在来看$w=z^n$的单叶性域.设$z_1=r_1\ee^{\ii\theta_1},z_2=r_2\ee^{\ii\theta_2}$,如果$z_1\ne z_2$,但$z_1^n=z_2^n$,即$r_1^n\ee^{\ii n\theta_1}=r_2^n\ee^{\ii n\theta_2}$,因而$r_1=r_2,\theta_1=\theta_2+\frac{2k\pi}n$.因此,只要域中不出现这样两个点,它们的辐角差等于$\frac{2\pi}n$,这样的域便是$w=z^n$的单叶性域.例如,$\bigg\{z:0<\arg z<\frac{2\pi}n\bigg\}$便是它的一个单叶性域.一般来说,域
\[\bigg\{z:\alpha<\arg z<\beta,0<\beta-\alpha\le\frac{2\pi}n\bigg\}\]
是它的单叶性域,它在$w=z^n$映射下的像是
\[\{w:n\alpha<\arg w<n\beta\}.\]

(2) $\mu=\frac1n,n${\kaishu 是一个自然数.}

$w=z^{\frac1n}$是$w=z^n$的反函数.因为对于一个给定的$z$,$z^{\frac1n}$有$n$个值,所以它是一个多值函数.由第 \ref{chap1} 章 \ref{sec1.2} 节知道,它的多值性也是由$\Arg z$的多值性产生的,所以$z=0$和$z=\infty$是它的支点.因而,在$\MC$去掉正实轴后所成的域上可以分出$n$个单值的全纯分支,它们是
\[w=\varphi_k(z)=\sqrt[\leftroot{-1}\uproot{2} n]{|z|}\bigg(\cos\frac{\theta+2k\pi}n+
\ii\sin\frac{\theta+2k\pi}n\bigg),k=0,1,\cdots,n-1.\]
这里,$\theta=\arg z$,它的变化范围是$0<\arg z<2\pi$. $k=0$的那一支称为它的主支,直接记为$w=\sqrt[\leftroot{-1}\uproot{2} n]z$.

现在来看它的主支的映射性质.容易看出,它把从原点发出的射线$\arg z=\theta$变为从原点发出的射线$\arg w=\frac\theta n$.由此可知,$w=\sqrt[\leftroot{-1}\uproot{2} n]z$把除去正实轴以后的全平面单叶地映为角状域$\bigg\{z:0<\arg z<\frac{2\pi}n\bigg\}$.例如,$w=\sqrt z,w=\sqrt[4]z$分别把除去正实轴的全平面单叶地映为上半平面和第一象限.

(3) $\mu=a+b\ii$,{\kaishu 是一个复数.}

一般的幂函数$w=z^\mu$定义为
\[w=z^\mu=\ee^{\mu\Log z},\]
显然,它是一个多值函数.用$\Log z$的表达式代入上式,可得
\begin{align*}
w=z^\mu&=\ee^{(a+b\ii)(\log|z|+\ii\arg z+2k\pi \ii)}\\
&=\ee^{a\log|z|-b(\arg z+2k\pi)}\ee^{\ii[b\log|z|+a(\arg z+2k\pi)]},k=0,\pm1,\cdots.
\end{align*}
\begin{enumerate}[label=(\alph*),left=0.85cm]
  \item \label{2.5.3.1} 若$b=0,a=n$是一个整数,这时$w=z^n$是一个单值函数.
  \item \label{2.5.3.2} 若$b=0,a=\frac pq$是一个有理数,不妨设$p<q$,这时
  \[w=z^\mu=z^{\frac pq}=|z|^{\frac pq}\ee^{\ii\frac pq(\arg z+2k\pi)}.\]
  当$k=0,1,\cdots,q-1$时,$z^{\frac pq}$有$q$个不同的值,因此是一个$q$值函数.
  \item \label{2.5.3.3} 若$b=0,a$是一个无理数,这时
  \[w=z^\mu=|z|^a\ee^{\ii a\arg z}\ee^{\ii2k\pi a}.\]
  因为$a$是无理数,不论$k$取什么整数值,都不能使$ka$为一整数,因此$z^\mu$是一个无穷值函数.
  \item \label{2.5.3.4} 若$b\ne0$,则$w=z^\mu$是一无穷值函数.
\end{enumerate}

总之,在上面的情况 \ref{2.5.3.2},\ref{2.5.3.3},\ref{2.5.3.4} 下,$z^\mu$都是一个多值函数.它的多值性是由$\Log z$的多值性引起的,因此$z=0$和$z=\infty$是它的支点,而且在$\Log z$可以分出单值全纯分支的域内,$z^\mu$也能分出单值全纯分支.设$\varphi_k(z)$是$\Log z$在域$D$中的单值全纯分支,$w_k(z)$是$z^\mu$的单值全纯分支,按定义,有
\[w_k(z)=\ee^{\mu\varphi_k(z)}.\]
其中
\[w_0(z)=\ee^{\mu\varphi_0(z)}=\ee^{\mu\log z},\]
称为$z^\mu$的主支. 因为$\varphi_k(z)$与$\varphi_{k+1}(z)$相差$2\pi\ii$,所以$w_k(z)$和$w_{k+1}(z)$相差$\ee^{2\mu\pi\ii}$. 由于$\varphi_k'(z)=\frac1z$,所以
\[w_k'(z)=\ee^{\mu\varphi_k(z)}\mu\varphi_k'(z)=\mu z^{\mu-1}=\mu\ee^{(\mu-1)\varphi_z(z)}.\]

有了上面这些知识,我们就可以作一些简单的保角变换了.

\begin{example}\label{exam2.4.5}
求以保角变换,把除去线段$\{z=a+\ii y:0<y<h\}$的上半平面变为上半平面.
\end{example}
\begin{solution}
初看起来,解这样的题目很困难,因为并没有一个现成的变换可以达到上述目的.我们的想法是把整个变换过程分解
\begin{figure}[!ht]
\centering
\begin{tikzpicture}[thick,>=Stealth]
\draw(-2,0)--(0,0)node[below]{$O$}--(0.8,0)node[below]{$a$}--(2,0)
(0.8,0)--(0.8,1)node[above]{$a+h\textrm i$};
\fill(0.8,1)circle(1pt)(0,0)circle(1pt);
\draw[->](1.5,0.6)--node[above]{$z_1=z-a$}(3.5,0.6);
\draw[->](6.5,0.6)--node[above]{$z_2=z_1^2$}(8.5,0.6);
\draw[->](-1,-1.4)--node[above]{$z_3=z_2+h^2$}(1,-1.4);
\draw[xshift=5cm,->](-1,-1.4)--node[above]{$w=\sqrt{z_3}$}(1,-1.4);
\begin{scope}[xshift=5cm]
\draw(-2,0)--(0,0)node[below]{$O$}--(2,0)(0,0)--(0,1)node[above]{$h\textrm i$};
\fill(0,1)circle(1pt);
\end{scope}
\begin{scope}[xshift=10cm]
\draw(-2,0)node[below]{$-h^2$}--(-1,0)node[below]{$O$}--(2,0);
\fill(-1,0)circle(1pt);
\fill(-2,0)circle(1pt);
\end{scope}
\draw(0,-2)node[below]{$O$}--(4,-2);
\fill(0,-2)circle(1pt);
\begin{scope}[xshift=6cm]
\draw(0,-2)--(2,-2)node[below]{$O$}--(4,-2);
\fill(2,-2)circle(1pt);
\end{scope}

\end{tikzpicture}
\caption{\label{fig2.5}}
\end{figure}
成若干个简单的步骤,而每一个步骤都可用我们已知的变换来实现,把这些变换复合起来,就是我们要找的变换.图 \ref{fig2.5} 就是整个变换的分解过程.所以,要找的变换就是
\begin{equation*}
w=\sqrt{z_3}=\sqrt{z_2+h^2}=\sqrt{z_1^2+h^2}=\sqrt{(z-a)^2+h^2}.
\end{equation*}
\end{solution}

\begin{example}
求一保角变换,把除去割线$\{z=x+\ii:-\infty<x<-1\}$后的带状域$\{z:0<\Im z<2\}$变为上半平面.
\end{example}
\begin{solution}
图 \ref{fig2.6} 是变换的分解过程.由此可见,要找的变换就是
\[w=\sqrt{\ee^{\pi z}+\ee^{-\pi}}.\]
这里,最后一个步骤用到了例 \ref{exam2.4.5} 的结果.
\end{solution}
\begin{figure}[!ht]
\centering
\begin{tikzpicture}[thick,>=Stealth,scale=0.8]
\draw(-3,0)--(0,0)node[below]{$O$}--(3,0);
\draw(-3,2)--(0,2)node[above]{$2\textrm i$}--(3,2);
\draw(-3,1)--(-1,1)node[above]{$-1+\textrm i$};
\fill(0,0)circle(1pt)(0,2)circle(1pt)(-1,1)circle(1pt);
\draw[->](4,pi/2)--node[above]{$z_1=\frac\pi2z$}(6,pi/2);
\begin{scope}[xshift=10cm,yscale=pi/2]
\draw(-3,0)--(0,0)node[below]{$O$}--(3,0);
\draw(-3,2)--(0,2)node[above]{$\pi\textrm i$}--(3,2);
\draw(-3,1)--(-1,1)node[above]{$-\frac\pi2+\frac\pi2\textrm i$};
\end{scope}
\fill[xshift=10cm](0,0)circle(1pt)(0,pi)circle(1pt)(-1,pi/2)circle(1pt);
\begin{scope}[yshift=-3cm]
\draw(-3,0)--(0,0)node[below]{$O$}--(3,0)(0,0)
--(0,1)node[above]{$\textrm e^{-\textrm i\frac\pi2}$};
\fill(0,0)circle(1pt)(0,1)circle(1pt);
\draw[->](-4,1)--node[above]{$z_2=\textrm e^{z_1}$}(-2,1);
\end{scope}
\draw[->,yshift=-3cm](3.5,pi/2)--node[above]{$w=\sqrt{z_2^2+\textrm e^{-\pi}}$}(6.5,pi/2);
\draw(7,-3)--(10,-3)node[below]{$O$}--(13,-3);
\fill(10,-3)circle(1pt);
\end{tikzpicture}
\caption{\label{fig2.6}}
\end{figure}
\subsection{三角函数}
如何定义复变数$z$的正弦函数$\sin z$和余弦函数$\cos z$呢?考虑的原则仍然是在讨论指数函数的定义时提出来的两条.由 Euler 公式知道
\begin{align*}
  &\ee^{\ii x}=\cos x+\ii\sin x,\\
  &\ee^{-\ii x}=\cos x-\ii\sin x,
\end{align*}
由此即得
\begin{align*}
  &\cos x=\frac12(\ee^{\ii x}+\ee^{-\ii x}),\\
  &\sin x=\frac1{2\ii}(\ee^{\ii x}-\ee^{-\ii x}).
\end{align*}
它启发我们给出$\sin z,\cos z$的下列定义:

设$z$是任意复数,定义
\begin{align*}
  &\cos z=\frac12(\ee^{\ii z}+\ee^{-\ii z}),\\
  &\sin z=\frac1{2\ii}(\ee^{\ii z}-\ee^{-\ii z}).
\end{align*}
下面是它们的一些主要性质:
\begin{eenum}\parindent=2em
  \item 因为$\ee^{\ii z},\ee^{-\ii z}$是整函数,所以$\cos z$和$\sin z$也都是整函数.
而且
\begin{align*}
  &(\cos z)'=-\sin z,\\
  &(\sin z)'=\cos z.
\end{align*}
\item 由于$\ee^{\ii z}$和$\ee^{-\ii z}$都以$2\pi$为周期,所以$\cos z$和$\sin z$也都以$2\pi$为周期.
\item $\cos z$是偶函数,$\sin z$是奇函数,即
\begin{align*}
  &\cos(-z)=\cos z,\\
  &\sin(-z)=-\sin z.
\end{align*}
\item 对任意复数$z_1$和$z_2$,有
\begin{align}
  &\cos(z_1+z_2)=\cos z_1\cos z_2-\sin z_1\sin z_2,\label{eq2.4.1}\\
  &\sin(z_1+z_2)=\sin z_1\cos z_2+\cos z_1\sin z_2.\label{eq2.4.2}
\end{align}
根据定义直接验证即得.
\item 在 \eqref{eq2.4.1} 式中令$z_1=z,z_2=-z$,即得
\[\cos^2 z+\sin ^2z=1.\]
在 \eqref{eq2.4.2} 式中令$z_1=z_2=z$,即得
\[\sin2z=2\sin z\cos z.\]
\item $\sin z$仅在$z=k\pi$处为零,$\cos z$仅在$z=k\pi+\frac\pi2$处为零,这里,$k=0,\pm1,\cdots$.这是因为
    \[\sin z=\frac1{2\ii}(\ee^{\ii z}-\ee^{-\ii z})=\frac1{2\ii\ee^{\ii z}}(\ee^{2\ii z}-1),\]
$\sin z=0$当且仅当$\ee^{2\ii z}-1=0$,而这只有当$z=k\pi$($k=0,\pm1,\pm2,\cdots$)时才能成立. 又因为$\cos z=\sin\bigg(\frac\pi2-z\bigg),\cos z=0$当且仅当$\sin\bigg(\frac\pi2-z\bigg)=0$,所以$z=\frac\pi2+k\pi$.

以上六条性质和实变数的正弦、余弦函数一样,但下面一条性质是不一样的:
\item $\cos z$和$\sin z$不是有界函数.

若取$z=\ii y,y$是实数,则
\[\cos z=\frac12(\ee^{\ii z}+\ee^{-\ii z})=\frac12(\ee^{-y}+\ee^y)\to\infty \mbox{ (当$y\to\infty$时)}.\]
对于$\sin z$,取$z=\frac\pi2+\ii y$,则有
\[\sin\bigg(\frac\pi2+\ii y\bigg)=\cos \ii y\to\infty\mbox{ (当$y\to\infty$时)}.\]
\end{eenum}

有了正弦、余弦函数,便可定义正切、余切函数:
\begin{align*}
  &\tan z=\frac{\sin z}{\cos z},\\
  &\cot z=\frac{\cos z}{\sin z}.
\end{align*}
前者在除掉$z=\frac\pi2+k\pi$($k=0,\pm1,\cdots$)的开平面上是全纯的,后者在除掉$z=k\pi$($k=0,\pm1,\cdots$)的开平面上全纯.

\subsection{多值函数\texorpdfstring{$w=\sqrt[\leftroot{-2}\uproot{6}n]{(z-a_1)^{\beta_1}\cdots(z-a_m)^{\beta_m}}$}{}}

这一小段我们讨论多值函数
\[w=\sqrt[\leftroot{-2}\uproot{6} n]{(z-a_1)^{\beta_1}\cdots(z-a_m)^{\beta_m}},\]
这里,$a_1,\cdots,a_m$是复数,$\beta_1,\cdots,\beta_m$是整数,$n$是正整数.在什么样的域中,从它能分出单值的全纯分支呢?任取$z_0\ne a_j,j=1,\cdots,m$,取充分小的简单闭曲线$\gamma_0$,使$z_0$在其内部,$a_1,\cdots,a_m$都在其外部.当$z$沿着$\gamma_0$的正方向走一圈时,$z-a_1,\cdots,z-a_m$的辐角都不变,故$z_0$不是支点.再看$a_j$;是不是支点,以$a_1$为例,记$z-a_j=r_j\ee^{\ii\theta_j},j=1,\cdots,m$,于是$w$可写为
\[w=\sqrt[\uproot{6}\leftroot{-2} n]{r_1^{\beta_1}\cdots r_m^{\beta_m}\ee^{\ii(\beta_1\theta_1+\cdots+\beta_m\theta_m)}}.\]
取简单闭曲线$\gamma_1$,使$a_1$在其内部,$a_2,\cdots,a_m$都在其外部。当$z$沿着$\gamma_1$的正方向走一圈时,$\theta_1$增加$2\pi$,$\theta_2,\cdots,\theta_m$都不变,$w$就变成
\[w=\sqrt[\leftroot{-2}\uproot{6}n]{r_1^{\beta_1}\cdots r_m^{\beta_m}\ee^{\ii(\beta_1\theta_1+\cdots+\beta_m\theta_m)+2\pi\beta_1\ii}}
=\ee^{\ii\frac{2\pi\beta_1}n}\sqrt[\leftroot{-2}\uproot{6}n]{r_1^{\beta_1}\cdots r_m^{\beta_m}\ee^{\ii(\beta_1\theta_1+\cdots+\beta_m\theta_m)}}.\]
因此,只有当$\beta_1$是$n$的倍数时,$w$的值才不变.其他$a_2,\cdots,a_m$点的情况也一样.于是得到结论:如果$\beta_j$不是$n$的倍数,那么$a_j$是它的支点.再看无穷远点,取充分大的圆周,使$a_1,\cdots,a_m$都在其内部.当$z$沿着这个圆周转一圈时,$z-a_1,\cdots,z-a_m$的辐角都要增加$2\pi$,$w$就变成
\[\ee^{\ii\frac{2\pi(\beta_1+\cdots+\beta_m)}n}\sqrt[\leftroot{-2}\uproot{6}n]{r_1^{\beta_1}\cdots r_m^{\beta_m}\ee^{\ii(\beta_1\theta_1+\cdots+\beta_m\theta_m)}}.\]
因而,只有当$\beta_1+\cdots+\beta_m$不是$n$的倍数时,$z=\infty$是支点.用同样的方法讨论,可以知道,如果简单闭曲线的内部包含$a_{j_1},\cdots,a_{j_r}$,与它们相应的和$\beta_{j_1}+\cdots+\beta_{j_r}$是$n$的倍数,那么当$z$沿该曲线转一圈后$w$的值不变. 根据这些考察,我们得到下面的
\begin{theorem}\label{thm2.4.7}
如果域$D$只包含这样的简单闭曲线,它的内部或者不含有任何支点,或者包含一组支点$a_{j_1},\cdots,a_{j_r}$,但与它们相应的和$\beta_ {j_1}+\cdots+\beta_{j_r}$是$n$的倍数,那么$w=\sqrt[\leftroot{-2}\uproot{6} n]{(z-a_1)^{\beta_1}\cdots(z-a_m)^{\beta_m}}$在$D$中能分出单值的全纯分支.
\end{theorem}

\begin{example}\label{exam2.4.8}
在怎样的域中,$w=\sqrt{z^2-1}$能分出单值的全纯分支?
\end{example}
\begin{solution}
由于
\[w=\sqrt{z^2-1}=\sqrt{(z-1)(z+1)},\]
这时$a_1=1,a_2=-1,\beta_1=\beta_2=1,n=2$. 所以$1$和$-1$都是它的支点,但无穷远点不是支点.因而,在除去线段$[-1,1]$的全平面(图 \ref{fig2.7})上,或者在除去两条割线$\{z:-\infty<z<-1\}$和$\{z:1<z<\infty\}$的全平面(图 \ref{fig2.8})上,都能分出单值的全纯分支.
\begin{figure}[!ht]
\centering
\begin{minipage}{0.48\linewidth}
\centering
\begin{tikzpicture}[thick,>=Stealth,]
\draw(-1,0)node[below]{$-1$}--(1,0)node[below]{$1$};
\fill(-1,0)circle(1pt)(1,0)circle(1pt);
\end{tikzpicture}
\caption{\label{fig2.7}}
\end{minipage}\hfill
\begin{minipage}{0.48\linewidth}
\centering
\begin{tikzpicture}[thick,>=Stealth,]
\draw(-3,0)--(-1,0)node[below]{$-1$}(1,0)node[below]{$1$}--(3,0);
\fill(-1,0)circle(1pt)(1,0)circle(1pt);
\end{tikzpicture}
\caption{\label{fig2.8}}
\end{minipage}
\end{figure}
\end{solution}

\begin{example}\label{exam2.4.9}
设$f(z)=\sqrt{z^{-1}(1-z)^3}(z+1)^{-1}$.试确定$f$在$[0,1]$的上岸取正值的单值全纯分支$f_0$,并计算$f_0(-\ii)$.
\end{example}
\begin{solution}
多值性主要发生在带根号的函数上,与$(z+1)^{-1}$无关.令$\varphi(z)=\sqrt{z^{-1}(1-z)^3}$,这时$z=0$和$z=1$都是$\varphi$的支点,但$z=\infty$不是.由定理  $\ref{thm2.4.7}$,$\varphi$能在除去线段$[0,1]$的全平面上分出单值全纯的分支.为了确定出在$[0,1]$上岸取正值的分支,记$z=r_1\ee^{\ii\theta_1},1-z=r_2\ee^{\ii\theta_2}$(图 \ref{fig2.9}),则
\[\sqrt{z^{-1}(1-z)^3}=\sqrt{r_1^{-1}r_2^3}
\ee^{\ii\big(\frac{3\theta_2-\theta_1}2+k\pi\big)},k=0,1.\]
\begin{figure}[!ht]
\centering
\begin{tikzpicture}[thick,>=Stealth,scale=2]
\draw[->](0,0)node[below left]{$O$}--(1,0)node[below left]{$1$}--(2,0);
\draw(0,0)--(0,-1)node[below]{$-\textrm i$};
\draw[->](0,0)--(35:1.4)node[above]{$z$}--(1,0)--([turn]0:0.8);
\fill(0,0)circle(0.5pt)(0,-1)circle(0.5pt)(1,0)circle(0.5pt)(35:1.4)circle(0.5pt);
\draw(0.15,0)arc(0:35:0.15)(1.1,0)arc(0:-100:0.1);
\draw(17:0.32)node{$\theta_1$}(1.15,-0.15)node{$\theta_2$};
\end{tikzpicture}
\caption{\label{fig2.9}}
\end{figure}
当$z$在$[0,1]$的上岸时,有
\[\theta_1=\theta_2=0,r_1=x,r_2=1-x.\]
显然,$k=0$的那一支在上岸取正值,记为$\varphi_0$,即
\[\varphi_0(z)=\sqrt{r_1^{-1}r_2^3}
\ee^{\ii\frac{3\theta_2-\theta_1}2}.\]

现在计算$\varphi_0(-\ii)$. 若让$z$从原点的左边到达$-\ii$,则
\[\theta_1=\frac32\pi,\theta_2=\frac\pi4,r_1=1,r_2=\sqrt2.\]
所以
\[\varphi_0(-\ii)=2^{\frac34}\ee^{-\frac{3\pi}8\ii},\]
故
\[f_0(-\ii)=\frac1{1-\ii}2^{\frac34}\ee^{-\frac{3\pi}8\ii}=2^{\frac14}\ee^{
-\frac\pi8\ii}.\]
若让$z$从$1$的右边到达$-\ii$,则
\[\theta_1=-\frac\pi2,\theta_2=-\frac74\pi,r_1=1,r_2=\sqrt2.\]
这时,
\[\varphi_0(-\ii)=2^{\frac34}\ee^{-\frac{19}8\pi\ii}=2^{\frac34}\ee^{-
\big(2\pi+\frac38\pi\big)}=2^{\frac34}\ee^{-\frac38\pi\ii}.\]
所得结果和刚才的完全一样.
\end{solution}
\begin{xiti}
\item 验证$\bar{\ee^z}=\ee^{\bar z}$.
\item 求$|\ee^{z^2}|$和$\arg \ee^{z^2}$.
\item 证明:若$\ee^z=1$,则必有$z=2k\pi\ii,k=0,\pm1,\cdots$.
\item 设$f$是整函数,$f(0)=1$. 证明:
\begin{enuma}
  \item 若$f'(z)=f(z)$对每个$z\in \MC$成立,则$f(z)\equiv \ee^z$;
  \item 若对每个$z,w\in\MC$,有$f(z+w)=f(z)f(w)$,且$f'(0)=1$,则$f(z)\equiv\ee^z$.
\end{enuma}
\item 试证:
\begin{enuma}
  \item $\bar{\Log z}=\Log \bar z,\forall z\in \MC\backslash\{0\}$;
  \item $\bar{\log z}=\log \bar z,\forall z\in \MC\backslash(-\infty,0]$.
\end{enuma}
\item 求$\Re(\log z^2)$和$\Im(\log z^2),z\in\MC\backslash\{0\}$.
\item 设$f$在$\MC\backslash(-\infty,0]$中全纯,$f(1)=0$. 证明:
\begin{enuma}
  \item 若$f'(z)=\ee^{-f(z)},z\in\MC\backslash(-\infty,0]$,则$f(z)\equiv \log z$;
  \item 若$f(zw)=f(z)+f(w),z\in\MC\backslash(-\infty,0],w\in(0,\infty)$,且$f'(1)=1$,则$f(z)\equiv\log z$.
\end{enuma}
\item 证明:$f(z)=z^2+2z+3$在$B(0,1)$中单叶.
\item 若$\frac pq$($q>0$)是有理数,证明:$z^{\frac pq}=\big(\sqrt[\leftroot{-1}\uproot{2}q]z\big)^p$.
\item 验证$\bar{z^\mu}=\bar z^\mu$.
\item 验证$z^{2\mu},(z^\mu)^2$和$(z^2)^\mu$是否相等,并说明理由.
\item 设$f$在$\MC\backslash(-\infty,0]$上全纯,$f(1)=1,\mu>0$.证明:
\begin{enuma}
  \item 若$f'(z)=\mu\frac{f(z)}z,z\in\MC\backslash(-\infty,0]$,则
  \[f(z)\equiv|z|^\mu\ee^{\ii\mu\arg z};\]
  \item 若$f(zw)=f(z)+f(w),z\in\MC\backslash(-\infty,0],w\in(0,\infty)$,且$f'(1)=\mu$,则
      \[f(z)\equiv |z|^\mu\ee^{\ii\mu\arg z}.\]
\end{enuma}
\item 验证$\bar{\sin z}=\sin\bar z,\bar{\cos z}=\cos\bar z$.
\item 证明:
\begin{enuma}
  \item $\cos(z+w)=\cos z\cos w-\sin z\sin w$;
  \item $\sin(z+w)=\sin z\cos w+\cos z\sin w$.
\end{enuma}
\item 称$\varphi(z)=\frac12\bigg(z+\frac1z\bigg)$为\textbf{Rokovsky函数}\index{R!Rokovsky函数}.证明下面四个域都是$\varphi$的单叶性域:
\begin{enuma}
  \item 上半平面$\{z\in\MC:\Im z>0\}$;
  \item 下半平面$\{z\in\MC:\Im z<0\}$;
  \item 无心单位圆盘$\{z\in\MC:0<|z|<1\}$;
  \item 单位圆盘的外部$\{z\in\MC:|z|>1\}$.
\end{enuma}
\item 求上题中的四个域在映射$\varphi(z)=\frac12\bigg(z+\frac1z\bigg)$下的像.
\item 证明下面三个域都是$\cos z$和$\sin z$的单叶性域:
\begin{enuma}
  \item 条形域$\{z\in\MC:\theta_0<\Re z<\theta_0+\pi\}$;
  \item 半条形域$\{z\in\MC:\theta_0<\Re z<\theta_0+2\pi,\Im z>0\}$;
  \item 半条形域$\{z\in\MC:\theta_0<\Re z<\theta_0+2\pi,\Im z<0\}$.
\end{enuma}
\item 证明:$w=\cos z$将半条形域
\[\{z\in\MC:0<\Re z<2\pi,\Im z>0\}\]
一一地映为$\MC\backslash[-1,\infty)$.
\item 证明:$w=\sin z$将半条形域
\[\bigg\{z\in\MC:-\frac\pi2<\Re z<\frac\pi2,\Im z>0\bigg\}\]
一一地映为上半平面.
\item 证明$B(0,1)$是$f(z)=\frac z{(1-z)^2}$的单叶性域,并求出$f\big(B(0,1)\big)$.
\item 当$z$按逆时针方向沿圆周$\{z\in\MC:|z|=2\}$旋转一圈后,计算下列函数辐角的增量:
\begin{tasks}(2)
  \task $(z-1)^{\frac12}$;
  \task $(1+z^4)^{\frac13}$;
  \task $(z^2+2z-3)^{\frac14}$;
  \task $\bigg(\frac{z-1}{z+1}\bigg)^{\frac12}$;
  \task $\bigg(\frac{z^2-1}{z^2+5}\bigg)^{\frac17}$.
\end{tasks}
\item 设$f(z)=\frac{z^{p-1}}{(1-z)^p},0<p<1$.证明:$f$能在域$D=\MC\backslash[0,1]$上选出单值的全纯分支.
\item 证明:$f(z)=\Log\bigg(\frac{z^2-1}z\bigg)$能在域
\[D=\MC\backslash\big((-\infty,-1]\cup[0,1]\big)\]
上选出单值的全纯分支.
\item 设单叶全纯映射$f$将域$D$一一地映为$G$,证明:$G$的面积为
\[\iint\limits_D|f'(z)|^2\dx\dy.\]
\item 设$f$是域$D$上的单叶全纯映射,$z=\gamma(t)$($\alpha\le t\le\beta$)是$D$中的光滑曲线. 证明:$w=f\big(\gamma(t)\big)$的长度为
    \[\int_\alpha^\beta\big|f'\big(\gamma(t)\big)\big|\,|\gamma'(t)|\textrm dt.\]
\item 设$D$是$z$平面上去掉线段$[-1,\ii],[1,\ii]$和射线$z=\ii t$($1\le t<\infty$)后所得的域,证明函数$\Log(1-z^2)$能在$D$上分出单值全纯分支. 设$f$是满足$f(0)=0$的那个分支,试计算$f(2)$的值.
\item 证明函数$\sqrt[\leftroot{-1}\uproot{2}4]{(1-z)^3(1+z)}$能在$\MC\backslash[-1,1]$上选出一个单值全纯分支$f$,满足$f(\ii)=\sqrt2\ee^{-\frac\pi8\ii}$.试计算$f(-\ii)$的值.
\end{xiti}

\section{分式线性变换\label{sec2.5}}
形如$w=T(z)=\frac{az+b}{cz+d}$的映射称为\textbf{分式线性变换}\index{B!变换!分式线性变换}或\textbf{Mobius变换}\index{B!变换!Mobius变换},其中,$a,b,c,d$是复常数,且满足$ad-bc\ne0$.很明显,如果$ad-bc=0$,则$T(z)$是一常数或无意义,我们排除这种情形.

若$c\ne0$,则除去点$z=-\frac dc$外,$T(z)$在$\MC$上是全纯的,而且
\[T'(z)=\frac{ad-bc}{(cz+d)^2}\ne0,\]
所以分式线性变换在$z\ne-\frac dc$处是保角变换.若$c=0$,则必$d\ne0$,这时$T(z)=Az+B\left(A=\frac ad,B=\frac bd\right)$,称为\textbf{整线性变换}\index{B!变换!整线性变换},它是一个整函数.

从方程$w=T(z)$中把$z$解出来,得
\[z=T^{-1}(w)=\frac{-dw+b}{cw-a},\]
称它为$w=T(z)$的逆变换,它仍然是一个分式线性变换. 由此可知,$w=T(z)$在$\MC$上是单叶的. 当$c\ne0$时,我们规定$T\bigg(-\frac dc\bigg)=\infty,T(\infty)=\frac ac$;当$c=0$时,规定$T(\infty)=\infty$. 于是,分式线性变换$w=T(z)$把$\MC_\infty$单叶地映为$\MC_\infty$.

设$S$和$T$是两个分式线性变换,那么它们的复合$S\circ T$也是分式线性变换,且对每一个$T$,有逆变换$T^{-1}$,即$T\big(T^{-1}(z)\big)=z$.所以,分式线性变换的全体在复合运算下构成一个群.

分式线性变换有一些有趣而重要的性质:

(1) {\kaishu 分式线性变换把圆周变为圆周.}

我们先考虑整线性变换$w=az+b$,若记$a=r\ee^{\ii\theta}$,则$w=r\ee^{\ii\theta}z+b$.容易看出,它可由下列三个简单的变换复合而成:
\begin{align*}
  &z'=\ee^{\ii\theta}z,\\
  &z''=rz',\\
  &w=z''+b.
\end{align*}
第一个是旋转变换,第二个是伸缩变换,第三个是平移变换。这里,每一个变换都把圆周变为圆周,因此整线性变换把圆周变为圆周.对于一般的分式线性变换,不妨设$c\ne0$,于是
\[w=\frac{az+b}{cz+d}=\frac ac+\frac{bc-ad}{c(cz+d)}.\]
若记$\alpha=\frac ac,\beta=\frac{bc-ad}c$,则上式可写为
\[w=\alpha+\frac{\beta}{cz+d}.\]
它由下列三个变换复合而成:
\begin{align*}
  &z'=cz+d,\\
  &z''=\frac1{z'},\\
  &w=\alpha+\beta z''.
\end{align*}
其中,有两个变换是整线性变换,它们都把圆周变为圆周.如果能证明$w=\frac 1z$也把圆周变为圆周,那么就证明了这一小段的标题上的结论.由于分式线性变换是定义在整个闭平面$\MC_\infty$上的,我们把直线看成是过无穷远点的圆周.于是,平面上的任一圆周都可写为(见习题 \hyperlink{xiti1.2}{1.2} 的第 \hyperlink{xiti1.2.14}{14} 题):
\begin{equation}\label{eq2.5.1}
az\bar z+\bar \beta z+\beta \bar z+d=0
\end{equation}
这里,$a,d$是实数,$\beta$是复数,且满足$|\beta|^2-ad>0$. 变换$w=\frac1z$把方程 \eqref{eq2.5.1} 变为
\[\frac a{w\bar w}+\frac{\bar \beta}w+\frac{\beta}{\bar w}+d=0,\]
即
\[dw\bar w+\bar\beta \bar w+\beta w+a=0,\]
它仍然是一个圆周.这样,我们已经证明了
\begin{theorem}\label{thm2.5.1}
  分式线性变换把圆周变成圆周.
\end{theorem}

现在的问题是,在把圆周变为圆周的同时,是否把圆的内部变成内部或外部?能否按预先的要求把内部变成内部或外部?下面将逐步解决这些问题.

(2) {\kaishu 交比是分式线性变换的不变量.}

分式线性变换看上去有四个参数,但实际上独立的参数只有三个,因此有理由提出这样的问题:在$z$平面和$w$平面上分别给定三个点$z_1,z_2,z_3$和$w_1,w_2,w_3$,是否一定能找到分式线性变换$w=T(z)$,使得$w_j=T(z_j),j=1,2,3$?为了证明这一事实,先证明
\begin{prop}\label{prop2.5.2}
分式线性变换$T$最多只有两个不动点,除非$T$是恒等变换,即$T(z)\equiv z$.
\end{prop}
\begin{proof}
  所谓不动点,是指满足$T(z)=z$的$z$.如果$z$是一个不动点,则有$\frac{az+b}{cz+d}=z$,即$z$满足
\[cz^2+(d-a)z-b=0.\]
这是一个二次方程,最多只有两个根,即$T$最多只有两个不动点,除非$T(z)\equiv z$.
\end{proof}

为了具体写出把三点映为三点的分式线性变换,我们引进交比的概念.
\begin{definition}\label{def2.5.3}
设$z_1,z_2,z_3,z_4$是给定的四个点,其中至少有三个点是不相同的,称比值
\[\frac{z_1-z_3}{z_1-z_4}\bigg/\frac{z_2-z_3}{z_2-z_4}\]
为这四个点的\textbf{交比}\index{J!交比},记为$(z_1,z_2,z_3,z_4)$.
\end{definition}
当这些点中有无穷远点时,我们规定
\begin{gather*}
(\infty,z_2,z_3,z_4)=\frac{z_2-z_4}{z_2-z_3},\quad
(z_1,\infty,z_3,z_4)=\frac{z_1-z_3}{z_1-z_4},\\
(z_1,z_2,\infty,z_4)=\frac{z_2-z_4}{z_1-z_4},\quad
(z_1,z_2,z_3,\infty)=\frac{z_1-z_3}{z_2-z_3}.
\end{gather*}

按照交比的定义,有
\[(z,z_2,z_3,z_4)=\frac{z-z_3}{z-z_4}\cdot\frac{z_2-z_4}{z_2-z_3},\]
它是一个分式线性变换.若把它记为$L(z)$,那么
\begin{align*}
  &L(z_2)=1,\\
  &L(z_3)=0,\\
  &L(z_4)=\infty.
\end{align*}
现在可以证明
\begin{theorem}\label{thm2.5.4}
  有一个而且只有一个分式线性变换把$\MC_\infty$上三个不同的点$z_2,z_3,z_4$映为事先给定的$\MC_\infty$上的三个点$w_2,w_3,w_4$.
\end{theorem}
\begin{proof}
  令$L(z)=(z,z_2,z_3,z_4)$,已知
  \begin{equation}\label{eq2.5.2}
    \begin{aligned}
    &L(z_2)=1,\\
    &L(z_3)=0,\\
    &L(z_4)=\infty.
    \end{aligned}
  \end{equation}
再令$S(z)=(z,w_2,w_3,w_4)$,则同样有
\begin{equation*}
    \begin{aligned}
    &S(w_2)=1,\\
    &S(w_3)=0,\\
    &S(w_4)=\infty.
    \end{aligned}
  \end{equation*}
于是
\begin{equation*}
  \begin{aligned}
    &S^{-1}(1)=w_2,\\
    &S^{-1}(0)=w_3,\\
    &S^{-1}(\infty)=w_4.
  \end{aligned}
\end{equation*}
若令$M=S^{-1}\circ L$,则
\[M(z_2)=S^{-1}\big(L(z_2)\big)=S^{-1}(1)=w_2.\]
同样道理,$M(z_3)=w_3,M(z_4)=w_4$.所以,$M$即为所求的分式线性变换.

现证唯一性.如果还有另外一个分式线性变换$M_1$,也满足$M_1(z_j)=w_j$,$j=2$,$3$,$4$,那么分式线性变换$M^{-1}\circ M_1$便有三个不动点$z_2,z_3,z_4$.根据命题  \ref{prop2.5.2},它只能是恒等变换,即$M^{-1}\big(M_1(z)\big)\equiv z$,于是$M_1(z)\equiv M(z)$.
\end{proof}

根据定理 \ref{thm2.5.4} 的证明,$w=M(z)=S^{-1}\big(L(z)\big)$便是把$z_2,z_3,z_4$映为$w_2,w_2,w_4$的分式线性变换,此即
\begin{equation}\label{eq2.5.3}
(w,w_2,w_3,w_4)=(z,z_2,z_3,z_4).
\end{equation}
由 \eqref{eq2.5.3} 式即可写出具体的变换. 从 \eqref{eq2.5.3} 式还可得到交比的一个重要性质:
\begin{theorem}\label{thm2.5.5}
交比是分式线性变换的不变量.这就是说,如果分式线性变换$T$把$z_1,z_2,z_3,z_4$映为$T(z_1),T(z_2),T(z_3),T(z_4)$,那么
\[(z_1,z_2,z_3,z_4)=\big(T(z_1),T(z_2),T(z_3),T(z_4)\big).\]
\end{theorem}
\begin{proof}
不妨设$z_2,z_3,z_4$是三个不同的点,令$T(z_j)=w_j,j=2,3,4$,则由定理 \ref{thm2.5.4} 知道,$T$就是由等式
\[(z,z_2,z_3,z_4)=(w,w_2,w_3,w_4)\]
所确定的分式线性变换. 若设$T(z_1)=w_1$,则必有
\[(z_1,z_2,z_3,z_4)=(w_1,w_2,w_3,w_4),\]
这就是要证明的.

若$z_2,z_3,z_4$中有两点相同,则等式显然成立.
\end{proof}

除了交比以外,分式线性变换还有没有其他的不变量呢?当然,交比的函数仍然是不变量。下面的定理断言,此外再没有其他的不变量了.
\begin{theorem}\label{thm2.5.6}
如果$f(z_1,z_2,z_3,z_4)$是分式线性变换下的不变量,即对任意分式线性变换$T$,都有
\begin{equation}\label{eq2.5.4}
f(z_1,z_2,z_3,z_4)=f\big(T(z_1),T(z_2),T(z_3),T(z_4)\big),
\end{equation}
那么$f$只能是交比$(z_1,z_2,z_3,z_4)$的函数.
\end{theorem}
\begin{proof}
证明很简单.令$L$是这样的线性变换:
\[L(z)=(z,z_2,z_3,z_4),\]
由 \eqref{eq2.5.2} 式知$L(z_2)=1,L(z_3)=0,L(z_4)=\infty$. 把它们代入 \eqref{eq2.5.4} 式,即得
\[f(z_1,z_2,z_3,z_4)=f\big((z_1,z_2,z_3,z_4),1,0,\infty\big),\]
这就是要证明的.
\end{proof}

现在利用交比这个不变量来讨论第一小段末尾提出的问题.为此,先证明
\begin{prop}\label{prop2.5.7}
四点$z_1,z_2,z_3,z_4$共圆的充要条件是
\begin{equation}\label{eq2.5.5}
\Im(z_1,z_2,z_3,z_4)=0.
\end{equation}
\end{prop}
\begin{proof}
如果$z_1,z_2,z_3,z_4$四点共圆,令$L(z)=(z,z_2,z_3,z_4)$,则$L(z_2)=1,L(z_3)=0,L(z_4)=\infty$.这说明分式线性变换$L$把$z_1,z_2,z_3,z_4$四点所在的圆周变成了实轴,因而
\[L(z_1)=(z_1,z_2,z_3,z_4)=\text{实数}.\]
这就是 \eqref{eq2.5.5} 式.

反之,如果 \eqref{eq2.5.5} 式成立,那么$(z_1,z_2,z_3,z_4)$等于某个实数$t$.由于$L^{-1}(1)=z_2,L^{-1}(0)=z_3,L^{-1}(\infty)=z_4$,所以分式线性变换$L^{-1}$把实轴变成由$z_2,z_3,z_4$所确定的圆周$\gamma$.因为$(z_1,z_2,z_3,z_4)=t$,即$L(z_1)=t$,于是$L^{-1}(t)=z_1$,所以$z_1\in\gamma$. 因而$z_1,z_2,z_3,z_4$四点共圆.
\end{proof}

根据命题 \ref{prop2.5.7},当且仅当点$z$在由$z_1,z_2,z_3$所确定的圆周$\gamma$上时才有$\Im(z,z_1,z_2$,\\$z_3)=0$,剩下不在圆周$\gamma$上的点$z$必使$\Im(z,z_1,z_2,z_3)>0$或$\Im(z,z_1,z_2,z_3)<0$.容易看出,圆内(或圆外)的点$z$必使$\Im(z,z_1,z_2,z_3)$保持定号.若不然,必在圆内有点$a$和$b$,使得
\begin{gather*}
\Im(a,z_1,z_2,z_3)>0,\\
\Im(b,z_1,z_2,z_3)<0.
\end{gather*}
但$\Im(z,z_1,z_2,z_3)$是$z$的连续函数,故必在线段$[a,b]$上有点$c$,使得$\Im(c,z_1,z_2,z_3)=0$,这是不可能的.

现在的问题是:当$z_1,z_2,z_3$给定时,究竟是圆内的点还是圆外的点使$\Im(z,z_1,z_2,z_3)$\\$>0$?这与$z_1,z_2,z_3$的走向有关.
\begin{definition}\label{def2.5.8}
设$\MC_\infty$上的圆周$\gamma$把平面分成$g_1$和$g_2$两个域,$z_1,z_2,z_3$是$\gamma$上有序的三个点.如果当我们从$z_1$走到$z_2$再走到$z_3$时,$g_1$和$g_2$分别在我们的左边和右边,就分别称$g_1$和$g_2$为$\gamma$关于走向$z_1,z_2,z_3$的左边和右边.
\end{definition}

例如,实轴关于走向$0,1,\infty$的左边是上半平面;虚轴关于走向$\ii,0,-\ii$的右边是左半平面$\{z:\Re z<0\}$;单位圆周$\{z:|z|=1\}$关于走向$1,\ii,-1$的左边是圆的内部,右边是圆的外部;单位圆关于走向$-\ii,-1,\ii$的左边是圆的外部,右边是圆的内部,等等.

下面用交比来刻画圆周关于走向$z_1,z_2,z_3$的左边和右边.

\begin{prop}\label{prop2.5.9}
设$z_1,z_2,z_3$是$\MC_\infty$中的圆周$\gamma$上有序的三个点,那么$\gamma$关于走向$z_1,z_2,z_3$右边和左边的点$z$分别满足
\[\Im(z,z_1,z_2,z_3)>0\]
和
\[\Im(z,z_1,z_2,z_3)<0.\]
\end{prop}
\begin{proof}
先设$\gamma$是以$a$为中心的圆周,不妨假定$z_1,z_2,z_3$是顺时针方向(图 \ref{fig2.10}),这时,$\gamma$关于走向$z_1,z_2,z_3$的右边就是圆的内部,左边是圆的外部.我们只需证明
\begin{equation}\label{eq2.5.6}
\begin{aligned}
  &\Im(a,z_1,z_2,z_3)>0,\\
  &\Im(\infty,z_1,z_2,z_3)<0
\end{aligned}
\end{equation}
就够了. 从图 \ref{fig2.10} 中显然可见$\bigg|\frac{z_2-a}{z_3-a}\bigg|=1$,且

\noindent\begin{minipage}[b]{0.75\textwidth}
\[\arg\bigg(\frac{z_2-a}{z_3-a}\bigg)=\arg(z_2-a)-\arg(z_3-a)=\theta,\]
因而有
\[\frac{z_2-a}{z_3-a}=\ee^{\ii\theta}.\]
由于
\[\arg\bigg(\frac{z_3-z_1}{z_2-z_1}\bigg)=\arg(z_3-z_1)-\arg(z_2-z_1)=-\frac\theta2,\]
\end{minipage}
\begin{minipage}[b]{0.25\textwidth}
\centering
\begin{tikzpicture}[thick,>=Stealth,scale=1.6,every node/.style={inner sep=1pt}]
\draw(0,0)coordinate(a)node[below right]{$a$}circle(1)(70:1)coordinate(z3)
node[above right]{$z_3$}(110:1)coordinate(z2)node[above left]{$z_2$}
(170:1)coordinate(z1)node[left]{$z_1$}(-150:0.7)coordinate(z)node[below=2pt]{$z$};
\draw(z1)--(z2)(z2)--(a)(a)--(z3)(z1)--(z3)(z)--(a);
\foreach \x in {a,z1,z2,z3,z}
\fill(\x)circle(0.7pt);
\draw(70:0.2)arc(70:110:0.2);
\node at(90:0.34){$\theta$};
\end{tikzpicture}
\captionof{figure}{\label{fig2.10}}
\end{minipage}\\
若记
\[\bigg|\frac{z_3-z_1}{z_2-z_1}\bigg|=r,\]
则
\[\frac{z_3-z_1}{z_2-z_1}=r\ee^{-\ii\frac\theta2}.\]
于是
\begin{align*}
\Im(a,z_1,z_2,z_3)&=\Im\bigg(\frac{a-z_2}{a-z_3}\cdot\frac{z_1-z_3}{z_1-z_2}\bigg)\\
&=\Im\bigg(\frac{z_2-a}{z_3-a}\cdot\frac{z_3-z_1}{z_2-z_1}\bigg)\\
&=\Im\big(r\ee^{\ii\frac\theta2}\big)=r\sin\frac\theta2>0.
\end{align*}
最后一个不等式成立是由于$0<\theta<2\pi$之故.同时
\[
\Im(\infty,z_1,z_2,z_3)=\Im\bigg(\frac{z_1-z_3}{z_1-z_2}\bigg)
=\Im\big(r\ee^{-\ii\frac\theta2}\big)
=-r\sin\frac\theta2<0.
\]
这就是要证的 \eqref{eq2.5.6} 式.

\noindent\begin{minipage}[b]{0.3\textwidth}
\centering
\begin{tikzpicture}[thick,>=Stealth,scale=1.5,every node/.style={inner sep=1pt}]
\draw(0,0)coordinate(z)node[right]{$z$}(-2,0)coordinate(z1)node[above left]{$z_1$}
(-1.2,0.8)coordinate(z2)node[above left]{$z_2$}(-0.3,1.7)coordinate(z3)node[above left]{$z_3$};
\draw(-2.2,-0.2)--(0,2)node[right]{$\gamma$};
\foreach \x in {z,z1,z2,z3}
\fill(\x)circle(0.7pt);
\draw(z)--(z1)(z)--(z2)(z)--(z3);
\clip(z2)--(z)--(z3);
\draw(0,0)circle(0.2);
\node at(120:0.35){$\theta$};
\end{tikzpicture}
\captionof{figure}{\label{fig2.11}}
\end{minipage}
\begin{minipage}[b]{0.7\textwidth}\parindent=2em
现再设$\gamma$是一条直线,$z_1,z_2,z_3$在$\gamma$上的位置如图 \ref{fig2.11} 所示.在$\gamma$关于走向$z_1,z_2,z_3$的右边任取点$z$,记$\bigg|\frac{z_2-z}{z_3-z}\bigg|=r$. 由于
\[\arg\frac{z_2-z}{z_3-z}=\arg(z_2-z)-\arg(z_3-z)=\theta,\]
因而
\[\frac{z_2-z}{z_3-z}=r\ee^{\ii\theta}\,\,\mbox{($0<\theta<\pi$)}.\]
\end{minipage}\\
因为$z_1,z_2,z_3$共线,故可记为
\[z_3-z_1=\rho(z_2-z_1)\;\mbox{($\rho>0$)}.\]
于是
\begin{align*}
\Im(z,z_1,z_2,z_3)&=\Im\bigg(\frac{z_2-z}{z_3-z}\cdot\frac{z_3-z_1}{z_2-z_1}\bigg)\\
&=\Im(\rho r\ee^{\ii\theta})=\rho r\sin\theta>0.
\end{align*}
最后一个不等式成立是因为$0<\theta<\pi$.对$\gamma$关于走向$z_1,z_2,z_3$左边的点,可同法证之.
\end{proof}


现在可以证明我们的主要结果:
\begin{theorem}\label{thm2.5.10}
设$\gamma_1$和$\gamma_2$是$\MC_\infty$中的两个圆周,$z_1,z_2,z_3$是$\gamma_1$上有序的三个点.如果分式线性变换$T$把$\gamma_1$映为$\gamma_2$,那么它一定把$\gamma_1$关于走向$z_1,z_2,z_3$的右边和左边分别变为$\gamma_2$关于走向$T(z_1),T(z_2),T(z_3)$的右边和左边.
\end{theorem}
\begin{proof}
记$\gamma_1$关于走向$z_1,z_2,z_3$的给右边为$g_1,\gamma_2$关于走向$T(z_1),T(z_2),T(z_3)$的右边为$g_2$. 任取$z\in g_1$,由命题 \ref{prop2.5.9} 知,$\Im(z,z_1,z_2,z_3)>0$. 由交比在分式线性变换下的不变性,可得
\[\Im\big(T(z),T(z_1),T(z_2),T(z_3)\big)=\Im(z,z_1,z_2,z_3)>0.\]
仍由命题 \ref{prop2.5.9} 知道$T(z)\in g_2$,因而$T(g_1)\subset g_2$. 反之,任取$w\in g_2$,则由命题 \ref{prop2.5.9} 知
\[\Im\big(w,T(z_1),T(z_2),T(z_3)\big)>0.\]
记$T^{-1}(w)=z$,即$w=T(z)$,于是
\[\Im(z,z_1,z_2,z_3)=\Im\big(T(z),T(z_1),T(z_2),T(z_3)\big)>0.\]
即$z\in g_1$,这就证明了$g_2\subset T(g_1)$. 因而$T(g_1)=g_2$.

关于左边的情形可同样证明.
\end{proof}

定理 \ref{thm2.5.10} 解决了本节第一小段末尾提出的问题.
\begin{example}\label{exam2.5.11}
求一分式线性变换,把月牙形域$D=\{z:|z|>1,|z-1|<2\}$变为带状域$G=\{w:0<\Re w<1\}$(见图 \ref{fig2.12}).
\end{example}
\begin{solution}
先设法把月牙形域$D$边界的两个圆周变为两条平行直线,同时把单位圆周变成虚轴,只要让$-1,\ii,1$分别变为$\infty,\ii,0$,即能达到目的.

这个变换可以用等式 \ref{eq2.5.3} 的办法来做,但下面的方法更简单. 因为$-1$变成$\infty$,$1$变成$0$,所以变换一定是$w=\lambda\frac{z-1}{z+1}$的形式,这里,$\lambda$是待定的常数. 再用$\ii$变成$\ii$带进去,得$\lambda=1$,故得$w=\frac{z-1}{z+1}$.令$z=3$,得$w=\frac12$,所以它把圆周$|z-1|=2$变为$\Re w=\frac12$. 又因为$1$变为$0$,所以它把$|z-1|<2$变为$\Re w<\frac12$. 因而,$w=\frac{z-1}{z+1}$把月牙形域$D$变为带状域$0<\Re w<\frac12$. 于是,变换$w=2\frac{z-1}{z+1}$即把$D$映为$G$.
\end{solution}
\begin{figure}[!ht]
\centering
\subcaptionbox{\label{fig2.12a}}[0.48\textwidth]
{
\begin{tikzpicture}[thick,>=Stealth,every node/.style={inner sep=1.5pt}]
\draw(0,0)node[below]{$O$}circle(1)(1,0)node[right]{$1$}circle(2)
(-1,0)node[left]{$-1$}(3,0)node[right]{$3$}(0,1)node[above]{$\textrm i$};
\fill(-1,0)circle(1pt)(1,0)circle(1pt)(0,1)circle(1pt)(0,0)circle(1pt)(3,0)circle(1pt);
\end{tikzpicture}
}
\subcaptionbox{\label{fig2.12b}}[0.48\textwidth]
{
\begin{tikzpicture}[thick,>=Stealth,every node/.style={inner sep=1.5pt}]
\draw(-1,0)--(0,0)node[below left]{$O$}--(1,0)node[below]{$\frac12$}
--(2,0)node[below right]{$1$}--(3,0);
\draw(0,-2)--(0,2)(2,-2)--(2,2);
\fill(0,0)circle(1pt)(1,0)circle(1pt)(2,0)circle(1pt);
\end{tikzpicture}
}
\caption{\label{fig2.12}}
\end{figure}


(3){\kaishu 对称点及其在分式线性变换下的不变性.}

先引进\textbf{对称点}\index{D!对称点}的概念:
\begin{definition}\label{def2.5.12}
设$\gamma$是以$a$为中心、以$R$为半径的圆周,如果点$z,z^\ast$在从$a$出发的射线上,且满足
\begin{equation}\label{eq2.5.7}
|z-a|\,|z^\ast-a|=R^2,
\end{equation}
则称$z,z^\ast$关于$\gamma$是对称的.如果$\gamma$是直线,则当$\gamma$是线段$[z,z^\ast]$
的垂直平分线时,称$z,z^\ast$关于$\gamma$是对称的.
\end{definition}

现设$z,z^\ast$关于圆周$\gamma$对称,因为它们位于从圆心$a$出发的射线上,所以
\[\arg(z-a)=\arg(z^\ast-a)=\theta,\]
因而
\begin{align*}
&z^\ast-a=|z^\ast-a|\ee^{\ii\theta},\\
&z-a=|z-a|\ee^{\ii\theta.}
\end{align*}
于是由 \eqref{eq2.5.7} 式,有
\[z^\ast-a=\frac{R^2}{|z-a|}\ee^{\ii\theta}=\frac{R^2}{|z-a|\ee^{-\ii\theta}}
=\frac{R^2}{\bar{z-a}},\]
故得
\[z^\ast=a+\frac{R^2}{\bar z-\bar a}.\]
由此可见,圆心和无穷远点是关于圆周的一对对称点.

若$\gamma$是过$a$且与实轴夹角为$\theta$的直线,设$z$和$z^\ast$关于$\gamma$对称(图 \ref{fig2.13}),则$|z-a|=|z^\ast-a|$,所以
\[z^\ast-a=|z^\ast-a|\ee^{\ii \alpha}=|z-a|\ee^{-\ii\beta}\ee^{\ii(\alpha+\beta)}
=(\bar z-\bar a)\ee^{2\ii\theta}.\]
因而有
\[z^\ast=a+(\bar z-\bar a)\ee^{2\ii\theta}.\]
\begin{figure}[!ht]
\centering
\begin{tikzpicture}[thick,>=Stealth,every node/.style={inner sep=1.5pt}]
\draw(0,0)coordinate(O)--(3,0)coordinate(A)--(6,0);
\draw(0,0)--(25:5)coordinate(z1)node[right]{$z^\ast$};
\draw(3,0)--(3.2,3)coordinate(z)node[above]{$z$};
\tkzInterLL(O,z1)(A,z)\tkzGetPoint{a}
\tkzDefMidPoint(z,z1)\tkzGetPoint{z2}
\draw(z2)--(a)node[above left]{$a$}--([turn]0:2.3)
(a)--(z2)--([turn]0:0.8)node[above]{$\gamma$};
\draw(0.3,0)arc(0:25:0.3);\node at(12:0.55){$\alpha$};
\tkzInterLL(a,z2)(O,A)\tkzGetPoint{B}
\begin{scope}
\clip(z)--(a)--(z2);
\draw(a)circle(0.3);
\node at(3.27,1.9){$\varphi$};
\end{scope}
\begin{scope}
\clip(a)--(B)--(A);
\draw(B)circle(0.2);
\node at(2.43,0.2){$\theta$};
\end{scope}
\begin{scope}
\clip(a)--(A)--(6,0);
\draw(A)circle(0.15);
\node at(3.25,0.28){$\beta$};
\end{scope}
\end{tikzpicture}
\caption{\label{fig2.13}}
\end{figure}

下面用交比给出两个点关于圆周对称的条件:
\begin{prop}\label{prop2.5.13}
设$\gamma$是$\MC_\infty$中的圆周,那么$z,z^\ast$关于$\gamma$对称的充要条件是对$\gamma$上任意三点$z_1,z_2,z_3$,有
\begin{equation}\label{eq2.5.8}
(z^\ast,z_1,z_2,z_3)=\bar{(z,z_1,z_2,z_3)}.
\end{equation}
\end{prop}
\begin{proof}
设$\gamma$是以$a$为中心、以$R$为半径的圆周,如果$z,z^\ast$关于$\gamma$对称,那么由交比在分式线性变换下的不变性,得
\begin{align*}
(z^\ast,z_1,z_2,z_3)&=\bigg(a+\frac{R^2}{\bar z-\bar a},z_1,z_2,z_3\bigg)\\
&=\bigg(\frac{R^2}{\bar z-\bar a},z_1-a,z_2-a,z_3-a\bigg)\\
&=\bigg(\bar z-\bar a,\frac{R^2}{z_1-a},\frac{R^2}{z_2-a},\frac{R^2}{z_3-a}\bigg)\\
&=(\bar z-\bar a,\bar z_1-\bar a,\bar z_2-\bar a,\bar z_3-\bar a)\\
&=(\bar z,\bar z_1,\bar z_2,\bar z_3)=\bar{(z,z_1,z_2,z_3)}.
\end{align*}

反之,如果 \eqref{eq2.5.8} 式成立,则可把上述推理过程倒推回去,即得
\[z^\ast=a+\frac{R^2}{\bar z-\bar a}.\]
这说明$z,z^\ast$关于$\gamma$是对称的.

对于直线的情形,可同法证之.
\end{proof}

在命题 \ref{prop2.5.13} 的基础上,可得对称点在分式线性变换下的不变性:
\begin{theorem}\label{thm2.5.14}
对称点在分式线性变换下不变. 这就是说,设分式线性变换$T$把圆周$\gamma$变为$\Gamma$,如果$z,z^\ast$是关于$\gamma$的对称点,那么$T(z),T(z^\ast)$是关于$\Gamma$的对称点.
\end{theorem}
\begin{proof}
在$\Gamma$上任取三点$w_1,w_2,w_3$,若记$z_j=T^{-1}(w_j),j=1,2,3$,则$z_1,z_2,z_3$是$\gamma$上的三个点.因$z,z^\ast$关于$\gamma$对称,由命题 \ref{prop2.5.13},得
\[(z^\ast,z_1,z_2,z_3)=\bar{(z,z_1,z_2,z_3)}.\]
再由交比在分式线性变换下的不变性,即得
\[\big(T(z^\ast),w_1,w_2,w_3\big)=\bar{\big(T(z),w_1,w_2,w_3\big)}.\]
最后由命题 \ref{prop2.5.13},即知$T(z)$和$T(z^\ast)$关于$\Gamma$对称.
\end{proof}

这一性质在作具体变换时非常有用.
\begin{example}\label{exam2.5.15}
求一分式线性变换,把上半平面变为单位圆的内部,但要求把上半平面中给定的点$a$变为圆心.
\end{example}
\begin{solution}
该分式线性变换一定把实轴变为单位圆周.由于$a$和$\bar a$关于实轴对称,由定理 \ref{thm2.5.14},这对点一定被变为一对关于单位圆周对称的点.已知$a$变为$0$,所以$\bar a$变为$\infty$.于是,这个变换可写成
\[w=\lambda\frac{z-a}{z-\bar a}.\]
为了使它把实轴变为单位圆周,$z=0$应变为单位圆周上的点,即$z=0$时应有$|w|=1$,由此得$\lambda=\ee^{\ii\theta}$.故所求的变换为
\[w=\ee^{\ii\theta}\frac{z-a}{z-\bar a}.\]
\end{solution}

\begin{example}\label{exam2.5.16}
求一分式线性变换,把单位圆的内部变成单位圆的内部,而且把圆内指定的点$a$变为圆心.
\end{example}
\begin{solution}
  因为$a$关于单位圆的对称点是$\frac1{\bar a}$,所以这个变换把$a$和$\frac1{\bar a}$分别变为$0$和$\infty$,故这个变换可写成
  \[w=\lambda\frac{z-a}{z-\frac1{\bar a}}=-\lambda\bar a\frac{z-a}{1-\bar az}=\mu\frac{z-a}{1-\bar az}.\]
为了把单位圆周变成单位圆周,即将满足$|z|=1$的$z$变为满足$|w|=1$的$w,\mu$必须满足
\[1=|w|=|\mu|\frac{|z-a|}{|1-\bar az|}=|\mu|\frac{|z-a|}{|z|\,|\bar z-\bar a|}=|\mu|,\]
即$\mu=\ee^{\ii\theta}$. 故所求的变换为
\[w=\ee^{\ii\theta}\frac{z-a}{1-\bar az}.\]
\end{solution}

这个变换十分重要,它是把单位圆盘一一地变为自己的变换,称为单位圆盘的\textbf{全纯自同构}\index{Q!全纯自同构}.以后我们将证明(定理 \ref{thm4.5.5}),把单位圆盘一一地变为自己的全纯映射只能是这种样子,再没有其他的变换.
\begin{example}\label{exam2.5.17}
求一分式线性变换,把偏心圆环$\{z:|z-3|>9,|z-8|<16\}$变为同心圆环$\bigg\{w:\frac23<|w|<1\bigg\}$(图 \ref{fig2.14}).
\begin{figure}[!ht]
\centering
\subcaptionbox{\label{fig2.14a}}[0.48\textwidth]
{
\begin{tikzpicture}[thick,>=Stealth,every node/.style={inner sep=1.5pt},scale=0.15]
\draw(3,0)circle(9);
\draw(8,0)circle(16);
\foreach \x in {3,8,-8,-24,0,12,24}
{\fill(\x,0)circle(5pt);}
\draw(0,0)node[below]{$O$}(12,0)node[below right]{$12$}(24,0)node[below right]{$24$}
(-8,0)node[below left]{$-8$}(3,0)node[below]{$3$}(8,0)node[below]{$8$}
(-24,0)node[below]{$-24$};
\end{tikzpicture}
}
\subcaptionbox{\label{fig2.14b}}[0.48\textwidth]
{
\begin{tikzpicture}[thick,>=Stealth,every node/.style={inner sep=1.5pt},scale=2]
\draw(0,0)circle(1)(0,0)circle(2/3);
\draw(0,0)node[below]{$O$}(2/3,0)node[right]{$\frac23$}(1,0)node[right]{$1$};
\foreach \x in{0,2/3,1}
{\fill(\x,0)circle(0.5pt);}
\end{tikzpicture}
}
\caption{\label{fig2.14}}
\end{figure}
\end{example}
\begin{solution}
如果能找到这对偏心圆的一对公共的对称点,则把它们变为$0$和$\infty$的分式线性变换一定把这对偏心圆变成一对以原点为中心的同心圆。由于这两个偏心圆的连心线在实轴上,所以这对对称点也必在实轴上.设它们为$x_1$和$x_2$,按对称点的定义,有
\[\left\{\begin{aligned}
&(x_1-3)(x_2-3)=81,\\
&(x_1-8)(x_2-8)=256.
\end{aligned}\right.\]
解之,得$x_0=0,x_2=-24$. 故可将变换写为
\[w=\lambda\frac z{z+24}.\]
让$|z-8|=16$变为$|w|=1$,取$z=24$,得$\lambda=2\ee^{\ii\theta}$.可以验证
\[w=\ee^{\ii\theta}\frac{2z}{z+24}\]
即为所求的变换.
\end{solution}

\begin{xiti}
\item 试求把上半平面映为上半平面的分式线性变换,使得$\infty,0,1$分别映为$0,1,\infty$.
\item 求出把上半平面映为单位圆盘的分式线性变换,使得$-1,0,1$分别映为$1,\ii,-1$.
\item 证明:分式线性变换$w=\frac{az+b}{cz+d}$的把上半平面映为上半平面的充要条件是$a,b,c,d$都是实数,而且$ad-bc>0$.
\item 试求把单位圆盘的外部$\{z:|z|>1\}$映为右半平面$\{w:\Re w>0\}$的分式线性变换,使得
\begin{enuma}
  \item $1,-\ii,-1$分别变为$\ii,0,-\ii$;
  \item $-\ii,\ii,1$分别变为$\ii,0,-\ii$.
\end{enuma}
\item 试求把上半平面映为自己的分式线性变换,使得实轴上的点$x_1,x_2,x_3$($x_1<x_2<x_3$)分别映为$0,1,\infty$.
\item 证明:$z_1,z_2$关于圆周$\bigg|\frac{z-z_1}{z-z_2}\bigg|=\lambda$($\lambda>0$)对称.\\
(\textbf{提示}:利用习题 \hyperlink{xiti1.2}{1.2} 中第 \hyperlink{xiti1.2.15}{15} 题的结果.)
\item 设$z_1,z_2$是关于圆周$\gamma=\{z:|z-a|=R\}$的一对对称点,证明:$\gamma$可以写成
\[\bigg|\frac{z-z_1}{z-z_2}\bigg|=\lambda\]
这种形式,其中,$\lambda=\frac R{|z_2-a|}=\frac{|z_1-a|}R$.
\item 设$z_1\ne z_2$,分式线性变换$w=T(z)=\frac{az+b}{cz+d}$满足$T(z_j)\ne\infty,j=1,2$.证明:$T$把圆周$\bigg|\frac{z-z_1}{z-z_2}\bigg|=\lambda$映为圆周
    \[\bigg|\frac{w-T(z_1)}{w-T(z_2)}\bigg|=\lambda\bigg|\frac{cz_2+d}{cz_1+d}\bigg|.\]
(\textbf{注意}:本题给出了圆周及其对称点在分式线性变换下的不变性的另一个证明.)
\item 证明:$z_1,z_2$关于圆周
\[az\bar z+\bar \beta z+\beta\bar z+d=0\]
对称的充要条件是
\[az_1\bar z_2+\bar \beta z_1+\beta\bar z_2+d=0.\]
\item 设$T(z)=\frac{az+b}{cz+d}$需是一个分式线性变换,如果记
\[\begin{pmatrix}
a&b\\c&d
\end{pmatrix}^{-1}=\begin{pmatrix}
\alpha&\beta\\\gamma&\delta
\end{pmatrix},\]
那么
\[T^{-1}(z)=\frac{\alpha z+\beta}{\gamma z+\delta}.\]
\item 设$T_1(z)=\frac{a_1z+b_1}{c_1z+d_1},T_2(z)=\frac{a_2z+b_2}{c_2z+d_2}$是两个分式线性变换,如果记
    \[\begin{pmatrix}
      a_1&b_1\\c_1&d_1
    \end{pmatrix}\begin{pmatrix}
    a_2&b_2\\c_2&d_2
    \end{pmatrix}=\begin{pmatrix}
    a&b\\c&d
    \end{pmatrix},\]
那么
\[(T_1\circ T_2)(z)=\frac{az+b}{cx+d}.\]
\item 设$\Gamma$是过$-1$和$1$的圆周,$z$和$w$都不在圆周上.如果$zw=1$,那么$z$和$w$必分别位于$\Gamma$的内部或外部.
\item 求一分式线性变换,把$B(0,1)$映为$B(0,1)$,且把$\frac12,2,\frac54+\frac34\ii$分别映为$\frac12,2,\infty$.
\item 求一单叶全纯映射,把沿线段$[0,1]$有割缝的单位圆盘映为上半平面.
\item 求一单叶全纯映射,把除去线段$[0,1+\ii]$的第一象限映为上半平面.
\item 求一单叶全纯映射,把半条形域
\[\bigg\{z:-\frac\pi2<\Re z<\frac\pi2,\Im z>0\bigg\}\]
映为上半平面,且把$\frac\pi2,-\frac\pi2,0$分别映为$1,-1,0$.
\item 求一单叶全纯映射,把除去线段$[a,a+h\ii]$的条形域
$\{z:0<\Im z<1\}$映为条形域$\{w:0<\Im w<1\}$,其中,$a$是实数,$0<h<1$.
\item 求一单叶全纯映射,把图 \ref{fig2.15} 所示的月牙形域映为$B(0,1)$.
\begin{figure}[!ht]
\centering
\begin{minipage}[b]{0.48\linewidth}
\centering
\begin{tikzpicture}[thick,>=Stealth,every node/.style={inner sep=1.5pt},scale=2.5]
\draw[->](-1,0)node[below]{$-1$}--(-1,1.2);
\draw[densely dashed](-1,0)--(0,0)node[below]{$O$}--(1,0)node[below]{$1$};
\draw(-1,0)--++(30:0.9);
\draw[out=30,in=150](-1,0)to(1,0);
\draw[xshift=-1cm](0,0)(30:0.2)arc(30:90:0.2);
\draw(-1,0)arc(180:0:1);
\node[above]at(0,1){$\textrm i$};
\foreach \x/\y in{-1/0,0/0,1/0,0/1}
\fill(\x,\y)circle(0.4pt);
\node at(-0.82,0.31){$\frac\pi3$};
\end{tikzpicture}
\caption{\label{fig2.15}}
\end{minipage}\hfill
\begin{minipage}[b]{0.48\linewidth}
\centering
\begin{tikzpicture}[thick,>=Stealth,every node/.style={inner sep=1.5pt},scale=2]
\draw(-0.5,0)circle(0.5)(0.5,0)circle(0.5);
\draw(-1,0)node[left]{$-1$}(0,0)node[right]{$O$}
(1,0)node[right]{$1$}(0,0)--(0,-2)node[right]{$-2\textrm i$};
\foreach \x/\y in {-1/0,0/0,1/0,0/-2}
\fill(\x,\y)circle(0.5pt);
\end{tikzpicture}
\caption{\label{fig2.16}}
\end{minipage}
\end{figure}
\item 求一单叶全纯映射,把除去线段$[1,2]$的单位园盘的外部映为上半平面.
\item 求一单叶全纯映射,把$B\bigg(-\frac12,\frac12\bigg)$和$B\bigg(\frac12,\frac12\bigg)$的外部除去线段$[-2\ii,0]$所成的域(见图 \ref{fig2.16})映为上半平面.
\item 设$0<r<a$,求一单叶全纯映射,把域
\[\{z\in\MC:\Re z>0,|z-a|>r\}\]
映为同心圆环
\[\{w\in\MC:\rho<|w|<1\}.\]
\end{xiti} 