

\chapter*{前\qquad 言}\addcontentsline{toc}{chapter}{前\qquad 言}


复变函数理论的基础是19世纪由三位杰出的数学家Cauchy,Weierstrass和Riemann奠定的,到现在已有一百多年的历史,这是一门相当成熟的学科.它在数学的其他分支(如常微分方程、积分方程、概率论、解析数论、算子理论及多复变函数论等)和自然科学的相关领域(如流体力学、空气动力学、电学及理论物理学等)中都有重要的应用.

复变函数论作为大学数学系的一门重要基础课,通常包含Cauchy的积分理论、Weierstrass的级数理论和Riemann的几何理论这三部分内容.本书作为这样一门课程的教材,就是以这三大块内容为中心来编写的,但在材料的取舍上与传统的教材略有不同.例如,在第 \ref{chap3} 章全纯函数的积分表示中,我们除了介绍全纯函数的Cauchy积分公式外,还对非全纯的函数(仅要求$f$的实部和虚部有一阶连续偏导数)建立了Cauchy积分公式,并用它得到一维$\bar\partial$问题的解,再利用这个解在第 \ref{chap5} 章中给出了Mittag-Leffler定理、Weierstrass因子分解定理和插值定理的证明,通过这些证明,使读者了解$\bar\partial$问题的解是构造全纯函数的重要工具,而这在以往的教材中是不被重视的.又如,在介绍调和函数理论(第 \ref{chap8} 章)的同时,我们还介绍了次调和函数的基本理论,因为次调和函数的理论在众多的其他数学分支中要遇到.再如,在本书的最后一章中介绍了多复变数全纯函数和全纯映射的一些基本性质.

在以往的教学中,曾有学生问:在微积分中,讲完单变量微积分,还要讲多变量微积分,为什么在复变函数课程中没有多变量函数的理论?这是一个很自然的问题,但回答起来并不容易.我们增添这样一章的目的,是要使学生了解单复变与多复变有许多本质的不同,在内容上和研究方法上都是如此.在多复分析已经成为数学研究的主流方向之一的今天,让学生们了解一些多复变最基本的知识是必要的.我们认为Riemann面属于另外一门课程的内容,很难在这样一本教材中说清楚,干脆就不提它了.至于个别定理的取舍,就不在这里一一介绍了.

书中定理的证明,大部分与传统的教材相似,只是在编排与叙述方式上有些差别,但也有若干创新之处.例如,在证明Weierstrass 关于级数的定理时,我们利用了全纯函数$f$的任意阶导数$f^{(n)}$在紧集$K$上的模可以用$f$在$K$的邻域上的模来控制这一事实,使证明得以简化,而且上述事实在别处还要用到.其他如边界对应定理和Weierstrass因子分解定理的证明,与传统的证明有更大的差别.

我们主张教材可以写得详细一些,教师不必都讲,给学生留一些自己学习的余地.例如,为了完整起见,在第 \ref{chap1} 章中我们比较详细地介绍了平面点集的知识,对于相当一部分学生来说,这部分内容在学多元函数的微积分时已经学过了,教师可不必再讲,留给学生备查就行了.又如,用残数理论计算定积分,我们介绍了不少方法,教师只需选择一部分来讲,其他可留作学生自学的材料.总之,教师应该根据实际情况作出取舍.

书中每节之后都附有不少习题,这是本书的重要组成部分.一些练习性的习题是为加深对教学内容的理解而设,学生都应该完成.一部分有一定难度的习题是为锻炼学生的综合分析能力而设,有些题初学时做不出来也不必介意,待学完本课程后回过头来还可以再想.

21世纪的钟声即将敲响,数学教学的改革已经提到人们的议事日程上.对于这样一门成熟的学科,应该如何改?我们认为,任何积极的改革都不应该触动前面提到的那三部分主要内容,而是应该在介绍这三部分主要内容的同时,尽可能使这门课程和现代数学更为接近。前面提到的$\bar\partial$问题及其应用、次调和函数和多复变基础知识等,都是为了这样一个目的而引进的.

作者曾在中国科学技术大学多次讲授这门课程,本书便是在这些讲稿的基础上写成的.在编写过程中,龚昇教授最近编著的《简明复分析》(北京大学出版社,1996年)对我们有很大启发,在此深致谢意.兄弟院校的一些优秀教材也对我们有很多帮助,在此一并致谢.

由于水平所限,书中缺点和错误在所难免,希望得到广大读者的批评指正.

\vspace*{1cm}
\hfill {\kaishu 史济怀\qquad 刘太顺}\hspace*{1.2cm}

\hfill 1998年2月于中国科学技术大学

\newpage
\chapter*{序\qquad 言}\addcontentsline{toc}{chapter}{序\qquad 言}
本书原作者为史济怀与刘太顺老师,其中史济怀老师是我在科大读本科期间三个学期的数学分析老师,可以说是相当幸运了,我们数院2012级本科班是史爷爷唯一教授了三个学期的班级. 后来我读大三的时候,史爷爷确实年龄太大,就彻底退出讲台了,自此我也就再也不曾见过史爷爷了.

复变函数一书是史爷爷所著的一本非常优秀的书籍,也是我本科学复分析时的授课教材,是一本非常不错的书,学校已经多年不再版,我决心为之重排. 本书是我在2019年年底,花费了大约两个月时间所完成的心血. 所有内容用 \LaTeX{} 重排,所有插图一律用 TikZ 重新绘制,排版方式尊重原书方正书版的排版格式,公式都是行显模式. 也非常感谢积分群几位群友帮助我核对其中很多排版的错误,同时我自己也更正了一些原书中的排版错误,当然,可能仍然存在一些其他的错误,欢迎各位的指正. 我将此书的 PDF 以及 \LaTeX{} 代码共享出来,真心希望能帮到有需要的同学. 其中 PDF 可以在我的博客 \href{yuxtech.github.io}{yuxtech.github.io} 获取,原代码我将会放在我的 github 主页\href{https://github.com/yuxtech}{https://github.com/yuxtech}. 如果有支持我的同学,请关注我的博客以及我的微信公众号: \textbf{向老师讲数学},在这里表示感谢.

\hfill {\kaishu 向禹}\hspace*{2.8cm}

\hfill 2020年9月于中国科学技术大学
